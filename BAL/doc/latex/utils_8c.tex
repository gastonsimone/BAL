\hypertarget{utils_8c}{
\subsection{Referencia del Archivo utils.c}
\label{utils_8c}\index{utils.c@{utils.c}}
}
Implementación de utilidades generales. 

{\tt \#include \char`\"{}utils.h\char`\"{}}\par
\subsubsection*{Funciones}
\begin{CompactItemize}
\item 
int \hyperlink{utils_8c_68d639af59cdb0c547839cb44aa06854}{binary\_\-search} (int $\ast$list, unsigned int size, int key)
\begin{CompactList}\small\item\em Busca un elemento en el array mediante bipartición. \item\end{CompactList}\item 
void \hyperlink{utils_8c_e92b7a18bc3552887809bbac1b732152}{insert\_\-sorted} (int $\ast$list, int n, int key)
\begin{CompactList}\small\item\em Inserta un elemento en un arreglo ordenado. \item\end{CompactList}\end{CompactItemize}


\subsubsection{Descripción detallada}
Implementación de utilidades generales. 



Definición en el archivo \hyperlink{utils_8c-source}{utils.c}.

\subsubsection{Documentación de las funciones}
\hypertarget{utils_8c_68d639af59cdb0c547839cb44aa06854}{
\index{utils.c@{utils.c}!binary\_\-search@{binary\_\-search}}
\index{binary\_\-search@{binary\_\-search}!utils.c@{utils.c}}
\paragraph{\setlength{\rightskip}{0pt plus 5cm}int binary\_\-search (int $\ast$ {\em list}, \/  unsigned int {\em size}, \/  int {\em key})}\hfill}
\label{utils_8c_68d639af59cdb0c547839cb44aa06854}


Busca un elemento en el array mediante bipartición. 

\begin{Desc}
\item[Parámetros:]
\begin{description}
\item[{\em list}]Arreglo de elementos \item[{\em size}]Tamaño de la lista \item[{\em key}]Elemento a buscar \end{description}
\end{Desc}
\begin{Desc}
\item[Devuelve:]El índice en donde se ubica el elemento en el arreglo, -1 si no fue encontrado\end{Desc}
Pos más información ver \href{http://es.wikipedia.org/wiki/B%C3%BAsqueda_dicot%C3%B3mica}{\tt http://es.wikipedia.org/wiki/B\%C3\%BAsqueda\_\-dicot\%C3\%B3mica} 

Definición en la línea 18 del archivo utils.c.

Referenciado por coord2cds(), mult\_\-mat\_\-cds(), y save\_\-cds().

\begin{Code}\begin{verbatim}19 {
20     int izq, der, medio;
21 
22     izq = 0;
23     der = size - 1;
24 
25     while (izq <= der)
26     {
27         medio = (int)((izq + der) / 2);
28         if (key == list[medio])
29             return medio;
30         else if (key > list[medio])
31             izq = medio + 1;
32         else
33             der = medio - 1;
34     }
35 
36     return -1;
37 }
\end{verbatim}
\end{Code}


\hypertarget{utils_8c_e92b7a18bc3552887809bbac1b732152}{
\index{utils.c@{utils.c}!insert\_\-sorted@{insert\_\-sorted}}
\index{insert\_\-sorted@{insert\_\-sorted}!utils.c@{utils.c}}
\paragraph{\setlength{\rightskip}{0pt plus 5cm}void insert\_\-sorted (int $\ast$ {\em list}, \/  int {\em n}, \/  int {\em key})}\hfill}
\label{utils_8c_e92b7a18bc3552887809bbac1b732152}


Inserta un elemento en un arreglo ordenado. 

\begin{Desc}
\item[Parámetros:]
\begin{description}
\item[{\em list}]Arreglo ordenado \item[{\em n}]Posición del último elemento del arreglo \item[{\em key}]elemento a insertar\end{description}
\end{Desc}
Inserta un elemento en un array ordenado preservando el orden. Esta función corresponde al loop interno de un {\em insertion sort\/}. Por más información ver \href{http://es.wikipedia.org/wiki/Ordenamiento_por_inserci%C3%B3n}{\tt http://es.wikipedia.org/wiki/Ordenamiento\_\-por\_\-inserci\%C3\%B3n}

\begin{Desc}
\item[Atención:]Esta función no solicita memoria. Se asume que la cantidad de memoria reservada es suficiente como para albergar al nuevo elemento. \end{Desc}


Definición en la línea 55 del archivo utils.c.

Referenciado por coord2cds(), y mult\_\-mat\_\-cds().

\begin{Code}\begin{verbatim}56 {
57     int pos = n;
58 
59     while (pos >= 0 && list[pos] > key) {
60         list[pos+1] = list[pos];
61         --pos;
62     }
63     list[pos+1] = key;
64 }
\end{verbatim}
\end{Code}


