\hypertarget{cholesky_8h}{
\subsection{Referencia del Archivo cholesky/cholesky.h}
\label{cholesky_8h}\index{cholesky/cholesky.h@{cholesky/cholesky.h}}
}
Archivo de cabecera para la factorización de Cholesky. 

{\tt \#include \char`\"{}../sparse/sp\_\-packcol.h\char`\"{}}\par
\subsubsection*{Funciones}
\begin{CompactItemize}
\item 
int $\ast$ \hyperlink{cholesky_8h_561de3be7ec3e6b00963f57f9c5ee5d5}{elimination\_\-tree} (\hyperlink{structsp__packcol}{sp\_\-packcol} $\ast$A, int $\ast$nnz)
\begin{CompactList}\small\item\em Calcula el árbol de eliminación de la matriz simétrica A. \item\end{CompactList}\item 
\hyperlink{structsp__packcol}{sp\_\-packcol} $\ast$ \hyperlink{cholesky_8h_0a79571a9a53e0dc2e4c90140e493108}{symbolic\_\-factorization} (\hyperlink{structsp__packcol}{sp\_\-packcol} $\ast$A)
\begin{CompactList}\small\item\em Factorización simbólica de la matriz A. \item\end{CompactList}\item 
void \hyperlink{cholesky_8h_2dd1cb8f4a30d7950399b658862a2180}{numerical\_\-factorization} (\hyperlink{structsp__packcol}{sp\_\-packcol} $\ast$A, \hyperlink{structsp__packcol}{sp\_\-packcol} $\ast$L)
\begin{CompactList}\small\item\em Sobreescribe {\tt L} con la factorización de Cholesky de A. \item\end{CompactList}\item 
void \hyperlink{cholesky_8h_700b53e08123405fc54742fb52098588}{cholesky\_\-Lsolver} (\hyperlink{structsp__packcol}{sp\_\-packcol} $\ast$L, double $\ast$b)
\begin{CompactList}\small\item\em Sobrescribe {\tt b} con la solución del sistema $Ly=b$. \item\end{CompactList}\item 
void \hyperlink{cholesky_8h_afba46e6cb519f5b021129cf06ece41e}{cholesky\_\-LTsolver} (\hyperlink{structsp__packcol}{sp\_\-packcol} $\ast$L, double $\ast$b)
\begin{CompactList}\small\item\em Sobreescribe {\tt b} con la solución de $L^Tx=b$. \item\end{CompactList}\item 
void \hyperlink{cholesky_8h_b61e2da86cddb63b2ef5d38b46a1d4ea}{cholesky\_\-solver} (\hyperlink{structsp__packcol}{sp\_\-packcol} $\ast$A, double $\ast$b)
\begin{CompactList}\small\item\em Resuelve un sistema lineal mediante el algoritmo de Cholesky optimizado para matrices dispersas. \item\end{CompactList}\end{CompactItemize}


\subsubsection{Descripción detallada}
Archivo de cabecera para la factorización de Cholesky. 

Este archivo define las funciones que implementan la factorización de Cholesky para matrices dispersas. Por más detalles acerca de la implementación, vea la documentación del archivo \hyperlink{cholesky_8c}{cholesky.c} 

Definición en el archivo \hyperlink{cholesky_8h-source}{cholesky.h}.

\subsubsection{Documentación de las funciones}
\hypertarget{cholesky_8h_700b53e08123405fc54742fb52098588}{
\index{cholesky.h@{cholesky.h}!cholesky\_\-Lsolver@{cholesky\_\-Lsolver}}
\index{cholesky\_\-Lsolver@{cholesky\_\-Lsolver}!cholesky.h@{cholesky.h}}
\paragraph{\setlength{\rightskip}{0pt plus 5cm}void cholesky\_\-Lsolver ({\bf sp\_\-packcol} $\ast$ {\em L}, \/  double $\ast$ {\em b})}\hfill}
\label{cholesky_8h_700b53e08123405fc54742fb52098588}


Sobrescribe {\tt b} con la solución del sistema $Ly=b$. 

\begin{Desc}
\item[Parámetros:]
\begin{description}
\item[{\em L}]Matriz {\tt nxn} triangular inferior, factorización de Cholesky \item[{\em b}]Vector de largo {\tt n} \end{description}
\end{Desc}
Al final de la ejecución el vector {\tt b} es igual al vector {\tt y} tal que $Ly=b$

\begin{Desc}
\item[Nota:]Ver el algoritmo 9.2 descrito en la sección 9.2 del paper de Stewart (ver las \hyperlink{index_refsec}{referencias}) \end{Desc}


Definición en la línea 329 del archivo cholesky.c.

Hace referencia a sp\_\-packcol::colp, sp\_\-packcol::ncol, sp\_\-packcol::rx, y sp\_\-packcol::val.

Referenciado por bal\_\-cholesky\_\-Lsolver(), y cholesky\_\-solver().

\begin{Code}\begin{verbatim}330 {
331     int ii, i, j;
332 
333     for (j = 0; j < L->ncol; ++j) {
334         b[j] /= L->val[L->colp[j]];
335 
336         for (ii = L->colp[j]+1; ii < L->colp[j+1]; ++ii) {
337             i = L->rx[ii];
338             b[i] -= ( b[j] * L->val[ii] );
339         }
340     }
341 }
\end{verbatim}
\end{Code}


\hypertarget{cholesky_8h_afba46e6cb519f5b021129cf06ece41e}{
\index{cholesky.h@{cholesky.h}!cholesky\_\-LTsolver@{cholesky\_\-LTsolver}}
\index{cholesky\_\-LTsolver@{cholesky\_\-LTsolver}!cholesky.h@{cholesky.h}}
\paragraph{\setlength{\rightskip}{0pt plus 5cm}void cholesky\_\-LTsolver ({\bf sp\_\-packcol} $\ast$ {\em L}, \/  double $\ast$ {\em b})}\hfill}
\label{cholesky_8h_afba46e6cb519f5b021129cf06ece41e}


Sobreescribe {\tt b} con la solución de $L^Tx=b$. 

\begin{Desc}
\item[Parámetros:]
\begin{description}
\item[{\em L}]Matriz {\tt nxn}triangular inferior, factorización de Cholesky \item[{\em b}]Vector de largo {\tt n} \end{description}
\end{Desc}
Al final de la ejecución el vector {\tt b} es igual al vector {\tt y} tal que $L^Ty=b$

\begin{Desc}
\item[Nota:]Ver el algoritmo 9.3 descrito en la sección 9.2 del paper de Stewart (ver las \hyperlink{index_refsec}{referencias}) \end{Desc}


Definición en la línea 354 del archivo cholesky.c.

Hace referencia a sp\_\-packcol::colp, sp\_\-packcol::ncol, sp\_\-packcol::rx, y sp\_\-packcol::val.

Referenciado por bal\_\-cholesky\_\-LTsolver(), y cholesky\_\-solver().

\begin{Code}\begin{verbatim}355 {
356     int ii, i, j;
357 
358     for (j = L->ncol-1; j >= 0; --j) {
359         for (ii = L->colp[j]+1; ii < L->colp[j+1]; ++ii) {
360             i = L->rx[ii];
361             b[j] -= ( b[i] * L->val[ii] );
362         }
363         b[j] /= L->val[L->colp[j]];
364     }
365 }
\end{verbatim}
\end{Code}


\hypertarget{cholesky_8h_b61e2da86cddb63b2ef5d38b46a1d4ea}{
\index{cholesky.h@{cholesky.h}!cholesky\_\-solver@{cholesky\_\-solver}}
\index{cholesky\_\-solver@{cholesky\_\-solver}!cholesky.h@{cholesky.h}}
\paragraph{\setlength{\rightskip}{0pt plus 5cm}void cholesky\_\-solver ({\bf sp\_\-packcol} $\ast$ {\em A}, \/  double $\ast$ {\em b})}\hfill}
\label{cholesky_8h_b61e2da86cddb63b2ef5d38b46a1d4ea}


Resuelve un sistema lineal mediante el algoritmo de Cholesky optimizado para matrices dispersas. 

\begin{Desc}
\item[Parámetros:]
\begin{description}
\item[{\em A}]Matriz del sistema lineal, simétrica definida positiva, dispersa empaquetada por columna. \item[{\em b}]ENTRADA/SALIDA: Vector a igualar mediante {\tt Ax}\end{description}
\end{Desc}
Esta función calcula un vetor {\tt x} tal que $Ax=b$ y lo guarda en el mismo vector {\tt b}. Esta implementación está completamente basada en el formato de matriz dispersa empaquetado por columna, para mejorar el desempeño de la aplicación.

La forma de proceder del algoritmo es la siguiente: \begin{enumerate}
\item Se realiza la factorización simbólica de {\tt A} \item Se realiza la factorización numérica de {\tt A} \item Se resuelven los sistemas triangulares $Ly=b$ y luego $L^Tx=y$ \end{enumerate}


Definición en la línea 384 del archivo cholesky.c.

Hace referencia a BAL\_\-ERROR, cholesky\_\-Lsolver(), cholesky\_\-LTsolver(), sp\_\-packcol::ncol, sp\_\-packcol::nrow, numerical\_\-factorization(), y symbolic\_\-factorization().

Referenciado por bal\_\-cholesky\_\-solver().

\begin{Code}\begin{verbatim}385 {
386     sp_packcol *L;
387 
388     if (A->nrow != A->ncol) {
389         /* Habría que verificar además que es simétrica y definida positiva */
390         BAL_ERROR("La matriz no es cuadrada");
391         return;
392     }
393 
394     L = symbolic_factorization(A);
395     numerical_factorization(A, L);
396     cholesky_Lsolver(L, b);
397     cholesky_LTsolver(L, b);
398 }
\end{verbatim}
\end{Code}


\hypertarget{cholesky_8h_561de3be7ec3e6b00963f57f9c5ee5d5}{
\index{cholesky.h@{cholesky.h}!elimination\_\-tree@{elimination\_\-tree}}
\index{elimination\_\-tree@{elimination\_\-tree}!cholesky.h@{cholesky.h}}
\paragraph{\setlength{\rightskip}{0pt plus 5cm}int$\ast$ elimination\_\-tree ({\bf sp\_\-packcol} $\ast$ {\em A}, \/  int $\ast$ {\em nnz})}\hfill}
\label{cholesky_8h_561de3be7ec3e6b00963f57f9c5ee5d5}


Calcula el árbol de eliminación de la matriz simétrica A. 

\begin{Desc}
\item[Parámetros:]
\begin{description}
\item[{\em A}]ENTRADA: Matriz simétrica dispersa empaquetada por columna a procesar \item[{\em nnz}]SALIDA: Cantidad de elementos no cero en la factorización de Cholesky \end{description}
\end{Desc}
\begin{Desc}
\item[Devuelve:]Estructura de padres del árbol de eliminación\end{Desc}
Esta función calcula el árbol de eliminación de la matriz A, devuelve la estructura de padres del árbol y la cantidad de elementos no cero de la factorización de Cholesky.

\begin{Desc}
\item[Nota:]Este paso es un paso previo a la factorización simbólica.

Esta implementación está basada en el algoritmo descrito en la sección 7.2 del paper de Stewart (ver las \hyperlink{index_refsec}{referencias}). \end{Desc}


Definición en la línea 42 del archivo cholesky.c.

Hace referencia a BAL\_\-ERROR, sp\_\-packcol::ncol, sp\_\-packcol::nrow, y row\_\-traversal\_\-packcol().

Referenciado por bal\_\-elimination\_\-tree(), y symbolic\_\-factorization().

\begin{Code}\begin{verbatim}43 {
44     int *touched, *parent;
45     int ix, i, j, posij, js;
46 
47     /* Inicialización */
48     *nnz = 0;
49 
50     if (A->nrow != A->ncol) {
51         BAL_ERROR("La matriz debe ser cuadrada");
52         return NULL;
53     }
54 
55     /* Inicializa estructuras */
56     touched = (int*)malloc(sizeof(int) * A->ncol);
57     parent = (int*)malloc(sizeof(int) * A->ncol);
58     for (ix = 0; ix < A->ncol; ++ix)
59         touched[ix] = parent[ix] = -1;
60 
61     /* Recorre las filas de A */
62     i = -2;
63     row_traversal_packcol(A, &i, &j, &posij);
64     for (ix = 0; ix < A->nrow; ++ix) {
65         while (row_traversal_packcol(A, &i, &j, &posij) != -1) {
66             if (i == j) {
67                 /* Procesar elemento de la diagonal */
68                 *nnz += 1;
69                 touched[j] = i;
70             }
71             else {
72                 /* Elemento no en la diagonal. Buscar el arbol */
73                 js = j;
74                 while (touched[js] != i) {
75                     touched[js] = i;
76                     *nnz += 1;
77 
78                     if (parent[js] == -1) {
79                         parent[js] = i;
80                         break;
81                     }
82                     
83                     js = parent[js];
84                 }
85             }
86         }
87     }
88 
89     /* Liberamos la memoria utilizada */
90     i = -2;
91     row_traversal_packcol(NULL, &i, &j, &posij);
92     free(touched);
93 
94     return parent;
95 }
\end{verbatim}
\end{Code}


\hypertarget{cholesky_8h_2dd1cb8f4a30d7950399b658862a2180}{
\index{cholesky.h@{cholesky.h}!numerical\_\-factorization@{numerical\_\-factorization}}
\index{numerical\_\-factorization@{numerical\_\-factorization}!cholesky.h@{cholesky.h}}
\paragraph{\setlength{\rightskip}{0pt plus 5cm}void numerical\_\-factorization ({\bf sp\_\-packcol} $\ast$ {\em A}, \/  {\bf sp\_\-packcol} $\ast$ {\em L})}\hfill}
\label{cholesky_8h_2dd1cb8f4a30d7950399b658862a2180}


Sobreescribe {\tt L} con la factorización de Cholesky de A. 

\begin{Desc}
\item[Parámetros:]
\begin{description}
\item[{\em A}]Matriz a factorizar con el algoritmo de Cholesky \item[{\em L}]Matriz pre inicializada donde guardar la factorización\end{description}
\end{Desc}
\begin{Desc}
\item[Nota:]La pre inicialización de la matriz {\tt L} se realiza con la función \hyperlink{cholesky_8c_372ebd05f16cc5771aacf2d9e2497119}{symbolic\_\-factorization()}.

Ver el algoritmo 9.1 descrito en la sección 9.1 del paper de Stewart (ver las \hyperlink{index_refsec}{referencias}). \end{Desc}


Definición en la línea 274 del archivo cholesky.c.

Hace referencia a sp\_\-packcol::colp, sp\_\-packcol::ncol, row\_\-traversal\_\-packcol(), sp\_\-packcol::rx, y sp\_\-packcol::val.

Referenciado por bal\_\-numerical\_\-factorization(), y cholesky\_\-solver().

\begin{Code}\begin{verbatim}275 {
276     int ii, i, j, poskj, kx, k;
277     double *accum, Lkj, Lkkinv;
278 
279     accum = (double*)malloc(sizeof(double) * L->ncol);
280 
281     k = -2;
282     row_traversal_packcol(L, &k, &j, &poskj);
283     for (kx = 0; kx < L->ncol; ++kx) {  /* Procesa la columna k */
284         while (row_traversal_packcol(L, &k, &j, &poskj) != -1) {
285             if (j == k) { /* Inicializar accum */
286                 for (ii = L->colp[k]; ii < L->colp[k+1]; ++ii)
287                     accum[L->rx[ii]] = 0;
288                 for (ii = A->colp[k]; ii < A->colp[k+1]; ++ii)
289                     accum[A->rx[ii]] = A->val[ii];
290             }
291             else { /* Restar L[k:n,j] de L[k:n,k] */
292                 Lkj = L->val[poskj];
293                 for (ii = poskj; ii < L->colp[j+1]; ++ii) {
294                     i = L->rx[ii];
295                     accum[i] -= ( Lkj * L->val[ii] );
296                 }
297             }
298         }
299 
300         /* Mueve L[k:n,k] de accum a L, ajustando sus componentes */
301         for (ii = L->colp[k]; ii < L->colp[k+1]; ++ii) { /* Corrección al algoritmo de Stewart */
302             i = L->rx[ii];
303             if (i == k) {
304                 L->val[ii] = sqrt(accum[i]);
305                 Lkkinv = 1 / L->val[ii];
306             }
307             else
308                 L->val[ii] = Lkkinv * accum[i];
309         }
310     }
311 
312     /* Libera memoria auxiliar */
313     k = -2;
314     row_traversal_packcol(NULL, &k, &j, &poskj);
315     free(accum);
316 }
\end{verbatim}
\end{Code}


\hypertarget{cholesky_8h_0a79571a9a53e0dc2e4c90140e493108}{
\index{cholesky.h@{cholesky.h}!symbolic\_\-factorization@{symbolic\_\-factorization}}
\index{symbolic\_\-factorization@{symbolic\_\-factorization}!cholesky.h@{cholesky.h}}
\paragraph{\setlength{\rightskip}{0pt plus 5cm}{\bf sp\_\-packcol}$\ast$ symbolic\_\-factorization ({\bf sp\_\-packcol} $\ast$ {\em A})}\hfill}
\label{cholesky_8h_0a79571a9a53e0dc2e4c90140e493108}


Factorización simbólica de la matriz A. 

\begin{Desc}
\item[Parámetros:]
\begin{description}
\item[{\em A}]Matriz a factorizar \end{description}
\end{Desc}
\begin{Desc}
\item[Devuelve:]La factorización simbólica de la matriz A\end{Desc}
\begin{Desc}
\item[Nota:]Ver el algoritmo 8.1 descrito en la sección 8.2 del paper de Stewart (ver las \hyperlink{index_refsec}{referencias}).\end{Desc}
\begin{Desc}
\item[Atención:]El pseudo-código mostrado en el paper de Stewart contiene errores. Se recomienda fuertemente comparar esta implementación (buscando los lugares marcados como corrección) con el pseudo-código de Stewart y ver las diferencias. \end{Desc}


Definición en la línea 188 del archivo cholesky.c.

Hace referencia a BAL\_\-ERROR, sp\_\-packcol::colp, elimination\_\-tree(), make\_\-column(), merge(), sp\_\-packcol::ncol, sp\_\-packcol::nnz, sp\_\-packcol::nrow, sp\_\-packcol::rx, y sp\_\-packcol::val.

Referenciado por bal\_\-symbolic\_\-factorization(), y cholesky\_\-solver().

\begin{Code}\begin{verbatim}189 {
190     int nnz;                /* Cantidad de elementos no cero */
191     int *parents;           /* Arbol de eliminacion */
192     int *bs;                /* Hijos por nodo ("Baby Sitter") */
193     int *ma;                /* "Merge array" */
194     sp_packcol *L;          /* Resultado de la factorizacion simbolica */
195     int i, j, k, jt;
196 
197     if (A->nrow != A->ncol) {
198         BAL_ERROR("La matriz debe ser cuadrada");
199         return NULL;
200     }
201 
202     /* Calcula la cantidad de elementos no cero */
203     parents = elimination_tree(A, &nnz);
204     free(parents);
205 
206     /* Inicializacion */
207     L = (sp_packcol*)malloc(sizeof(sp_packcol));
208     L->nrow = A->nrow;
209     L->ncol = A->ncol;
210     L->nnz = (unsigned)nnz;
211     L->colp = (unsigned int*)malloc(sizeof(unsigned int) * (A->ncol+1));
212     L->rx = (unsigned int*)malloc(sizeof(unsigned int) * nnz);
213     L->val = (double*)malloc(sizeof(double) * nnz);
214     for (i = 0; i <= A->ncol; ++i)
215         L->colp[i] = 0;
216     for (i = 0; i < nnz; ++i) {
217         L->rx[i] = 0;
218         L->val[i] = 0;
219     }
220 
221     bs = (int*)malloc(sizeof(int) * A->ncol);
222     ma = (int*)malloc(sizeof(int) * A->ncol);
223     for (i = 0; i < A->ncol; ++i) {
224         bs[i] = -1;
225         ma[i] = A->ncol;
226     }
227 
228     /* Itera en las columnas de A */
229     for (k = 0; k < A->ncol; ++k) {
230         /* Computar la estructura de la k-esima columna */
231         merge(A, k, k, ma);
232         j = bs[k];
233         bs[k] = -1; /* Corrección al algoritmo de Stewart */
234         while (j != -1) {
235             merge(L, j, k, ma);
236             jt = bs[j];
237             bs[j] = -1;
238             j = jt;
239         }
240 
241         /* Establecer la k-esima columna en L */
242         make_column(k, ma, L);
243 
244         /* Actualizar bs */
245         if (k != j) {
246             j = L->rx[L->colp[k] + 1];      /* j es el padre de k */
247             while (j != -1) {
248                 jt = j;
249                 j = bs[j];
250             }
251             bs[jt] = k;
252         }
253     }
254 
255     /* Libera memoria auxiliar */
256     free(bs);
257     free(ma);
258 
259     return L;
260 }
\end{verbatim}
\end{Code}


