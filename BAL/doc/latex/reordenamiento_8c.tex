\hypertarget{reordenamiento_8c}{
\subsection{Referencia del Archivo cholesky/reordenamiento.c}
\label{reordenamiento_8c}\index{cholesky/reordenamiento.c@{cholesky/reordenamiento.c}}
}
Implementación de algoritmos de reordenamiento. 

{\tt \#include $<$stdio.h$>$}\par
{\tt \#include $<$stdlib.h$>$}\par
{\tt \#include \char`\"{}reordenamiento.h\char`\"{}}\par
{\tt \#include \char`\"{}../sparse/sp\_\-packcol.h\char`\"{}}\par
\subsubsection*{Funciones}
\begin{CompactItemize}
\item 
\hyperlink{structsp__packcol}{sp\_\-packcol} $\ast$ \hyperlink{reordenamiento_8c_05f0067f3214ea7f66d4519c09b5c5c2}{permutar\_\-packcol} (unsigned int $\ast$p, \hyperlink{structsp__packcol}{sp\_\-packcol} $\ast$A)
\begin{CompactList}\small\item\em Aplica una permutación de columnas sobre la matriz {\tt A}. \item\end{CompactList}\item 
unsigned int $\ast$ \hyperlink{reordenamiento_8c_a1a7373269a886624a8371feb7bf6066}{reordenar\_\-packcol} (\hyperlink{structsp__packcol}{sp\_\-packcol} $\ast$A)
\begin{CompactList}\small\item\em Reordena las columnas de A para minimizar el {\em fill-in\/} producido por una etapa posterior de factorización. \item\end{CompactList}\end{CompactItemize}


\subsubsection{Descripción detallada}
Implementación de algoritmos de reordenamiento. 



Definición en el archivo \hyperlink{reordenamiento_8c-source}{reordenamiento.c}.

\subsubsection{Documentación de las funciones}
\hypertarget{reordenamiento_8c_05f0067f3214ea7f66d4519c09b5c5c2}{
\index{reordenamiento.c@{reordenamiento.c}!permutar\_\-packcol@{permutar\_\-packcol}}
\index{permutar\_\-packcol@{permutar\_\-packcol}!reordenamiento.c@{reordenamiento.c}}
\paragraph{\setlength{\rightskip}{0pt plus 5cm}{\bf sp\_\-packcol} $\ast$ permutar\_\-packcol (unsigned int $\ast$ {\em p}, \/  {\bf sp\_\-packcol} $\ast$ {\em A})}\hfill}
\label{reordenamiento_8c_05f0067f3214ea7f66d4519c09b5c5c2}


Aplica una permutación de columnas sobre la matriz {\tt A}. 

\begin{Desc}
\item[Parámetros:]
\begin{description}
\item[{\em p}]Permutación. {\tt p\mbox{[}i\mbox{]}} = La p\mbox{[}i\mbox{]}-ésima columna pasa a ser la columna i \item[{\em A}]Matriz a permutarle las columnas \end{description}
\end{Desc}
\begin{Desc}
\item[Devuelve:]La misma matriz {\tt A} pero con las matrices permutadas según {\tt p} \end{Desc}
\begin{Desc}
\item[Nota:]La matriz {\tt A} y el resultado no comparten memoria. \end{Desc}


Definición en la línea 21 del archivo reordenamiento.c.

Hace referencia a sp\_\-packcol::colp, sp\_\-packcol::ncol, sp\_\-packcol::nnz, sp\_\-packcol::nrow, sp\_\-packcol::rx, y sp\_\-packcol::val.

Referenciado por bal\_\-permutar\_\-packcol().

\begin{Code}\begin{verbatim}22 {
23     unsigned int i, j, col;
24     sp_packcol* B;
25 
26     B = (sp_packcol*)malloc(sizeof(sp_packcol));
27     B->nrow = A->nrow;
28     B->ncol = A->ncol;
29     B->nnz = A->nnz;
30 
31     if (B->nnz == 0) {
32         B->colp = NULL;
33         B->rx = NULL;
34         B->val = NULL;
35     }
36     else {
37         B->colp = (unsigned int*)malloc(sizeof(unsigned int) * (B->ncol+1));
38         B->rx = (unsigned int*)malloc(sizeof(unsigned int) * B->nnz);
39         B->val = (double*)malloc(sizeof(double) * B->nnz);
40 
41         j = 0;
42         for(col=0; col < B->ncol; ++col) {  /*< Para cada columna col en B */
43             B->colp[col] = j;                   /*< Los elementos de la columna col comenzarán en la entrada j */
44             for (i=A->colp[p[col]]; i < A->colp[p[col]+1]; ++i) {   /*< Recorre la columna p[col]-ésima de A */
45                 B->rx[j] = A->rx[i];                                /*< y copia el contenido */
46                 B->val[j] = A->val[i];
47                 ++j;
48             }
49         }
50         B->colp[col] = j;                   /*< Elemento extra para indicar el fin de la ultima columna */
51     }
52 
53     return B;
54 }
\end{verbatim}
\end{Code}


\hypertarget{reordenamiento_8c_a1a7373269a886624a8371feb7bf6066}{
\index{reordenamiento.c@{reordenamiento.c}!reordenar\_\-packcol@{reordenar\_\-packcol}}
\index{reordenar\_\-packcol@{reordenar\_\-packcol}!reordenamiento.c@{reordenamiento.c}}
\paragraph{\setlength{\rightskip}{0pt plus 5cm}unsigned int $\ast$ reordenar\_\-packcol ({\bf sp\_\-packcol} $\ast$ {\em A})}\hfill}
\label{reordenamiento_8c_a1a7373269a886624a8371feb7bf6066}


Reordena las columnas de A para minimizar el {\em fill-in\/} producido por una etapa posterior de factorización. 

\begin{Desc}
\item[Parámetros:]
\begin{description}
\item[{\em A}]Matriz a reordenar\end{description}
\end{Desc}
\begin{Desc}
\item[\hyperlink{todo__todo000002}{Tareas Pendientes}]Implementar un algoritmo de reordenamiento\end{Desc}
Actualmente esta función no implementa ningún algoritmo de reordenamiento. Simplemente retorna la permutación trivial para la cantidad de columnas de {\tt A}.

Un algoritmo de reordenamiento básicamente calcula una permutación a ser aplicada en las filas/columnas de la matriz.

Mientras que no se cuente con una implementación de un algoritmo de reordenamiento, se puede calcular una permutación mediante código externo y utilizar la función \hyperlink{reordenamiento_8c_05f0067f3214ea7f66d4519c09b5c5c2}{permutar\_\-packcol()}, que recibe una permutación como parámetro y realiza la permutación sobre una matriz empaquetada por columna.

\begin{Desc}
\item[Nota:]Para aquel que pretenda implementar un algoritmo de reordenamiento para BAL, una posibilidad es el algoritmo de Cuthill-McKee. Puede realizarse una implementación de cero, o tratar de reutilizar una existente, como la disponible en el siguiente sitio: \href{http://www.math.temple.edu/~daffi/software/rcm/}{\tt http://www.math.temple.edu/$\sim$daffi/software/rcm/} \end{Desc}


Definición en la línea 79 del archivo reordenamiento.c.

Hace referencia a sp\_\-packcol::ncol.

\begin{Code}\begin{verbatim}80 {
81     unsigned int *p;
82     unsigned int i;
83 
84     p = (unsigned int*)malloc(sizeof(unsigned int) * A->ncol);
85 
86     for (i=0; i < A-> ncol; ++i)
87         p[i] = i;
88 
89     return p;
90 }
\end{verbatim}
\end{Code}


