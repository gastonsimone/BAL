\hypertarget{sp__coord_8c}{
\subsection{Referencia del Archivo sparse/sp\_\-coord.c}
\label{sp__coord_8c}\index{sparse/sp\_\-coord.c@{sparse/sp\_\-coord.c}}
}
Archivo de implementación para matriz dispersa, formato simple. 

{\tt \#include $<$stdlib.h$>$}\par
{\tt \#include $<$stdio.h$>$}\par
{\tt \#include \char`\"{}sp\_\-coord.h\char`\"{}}\par
{\tt \#include \char`\"{}../utils.h\char`\"{}}\par
\subsubsection*{Funciones}
\begin{CompactItemize}
\item 
\hyperlink{structsp__coord}{sp\_\-coord} $\ast$ \hyperlink{sp__coord_8c_627349c38a6e7dc3792abd416651fb83}{mat2coord} (int n, int m, double $\ast$$\ast$mat)
\begin{CompactList}\small\item\em Genera la matriz dispersa equivalente a la matriz completa mat nxm. \item\end{CompactList}\item 
\hyperlink{structsp__coord}{sp\_\-coord} $\ast$ \hyperlink{sp__coord_8c_d6a1042d41ff43da1f26396f5803daac}{load\_\-coord} (FILE $\ast$fp)
\begin{CompactList}\small\item\em Lee una matriz en formato simple por coordenadas desde archivo. \item\end{CompactList}\item 
void \hyperlink{sp__coord_8c_65d8ffec5eb8263159ca9e6310c6e762}{save\_\-coord} (FILE $\ast$fp, \hyperlink{structsp__coord}{sp\_\-coord} $\ast$A)
\begin{CompactList}\small\item\em Escribe en {\tt fp} la matriz {\tt A} en un formato entendible por \hyperlink{sp__coord_8c_d6a1042d41ff43da1f26396f5803daac}{load\_\-coord()}. \item\end{CompactList}\item 
void \hyperlink{sp__coord_8c_ff646304bb38349c4620069c73a35760}{sp\_\-imprimir\_\-coord} (FILE $\ast$fp, \hyperlink{structsp__coord}{sp\_\-coord} $\ast$mat)
\begin{CompactList}\small\item\em Imprime la matriz guardada en formato simple por coordenadas en fp. \item\end{CompactList}\item 
void \hyperlink{sp__coord_8c_33dd472b98444c5d08c9099556226816}{free\_\-coord} (\hyperlink{structsp__coord}{sp\_\-coord} $\ast$A)
\begin{CompactList}\small\item\em Borra toda la memoria reservada por la matriz A. \item\end{CompactList}\end{CompactItemize}


\subsubsection{Descripción detallada}
Archivo de implementación para matriz dispersa, formato simple. 

Este archivo contiene la implementación de las funciones de utilidad para el formato de matriz dispersa simple. 

Definición en el archivo \hyperlink{sp__coord_8c-source}{sp\_\-coord.c}.

\subsubsection{Documentación de las funciones}
\hypertarget{sp__coord_8c_33dd472b98444c5d08c9099556226816}{
\index{sp\_\-coord.c@{sp\_\-coord.c}!free\_\-coord@{free\_\-coord}}
\index{free\_\-coord@{free\_\-coord}!sp_coord.c@{sp\_\-coord.c}}
\paragraph{\setlength{\rightskip}{0pt plus 5cm}void free\_\-coord ({\bf sp\_\-coord} $\ast$ {\em A})}\hfill}
\label{sp__coord_8c_33dd472b98444c5d08c9099556226816}


Borra toda la memoria reservada por la matriz A. 

Esta función libera toda la memoria reservada por la estructura de datos \hyperlink{structsp__coord}{sp\_\-coord}. 

Definición en la línea 206 del archivo sp\_\-coord.c.

Hace referencia a sp\_\-coord::cx, sp\_\-coord::rx, y sp\_\-coord::val.

Referenciado por bal\_\-free\_\-coord(), y load\_\-coord().

\begin{Code}\begin{verbatim}207 {
208     free(A->rx);
209     free(A->cx);
210     free(A->val);
211     free(A);
212 }
\end{verbatim}
\end{Code}


\hypertarget{sp__coord_8c_d6a1042d41ff43da1f26396f5803daac}{
\index{sp\_\-coord.c@{sp\_\-coord.c}!load\_\-coord@{load\_\-coord}}
\index{load\_\-coord@{load\_\-coord}!sp_coord.c@{sp\_\-coord.c}}
\paragraph{\setlength{\rightskip}{0pt plus 5cm}{\bf sp\_\-coord} $\ast$ load\_\-coord (FILE $\ast$ {\em fp})}\hfill}
\label{sp__coord_8c_d6a1042d41ff43da1f26396f5803daac}


Lee una matriz en formato simple por coordenadas desde archivo. 

\begin{Desc}
\item[Parámetros:]
\begin{description}
\item[{\em fp}]Archivo del cual leer la matriz \end{description}
\end{Desc}
\begin{Desc}
\item[Devuelve:]La matriz leída en formato simple por coordenadas\end{Desc}
Lee el archivo {\tt fp} en busca de una definición de matriz en formato simple por coordenadas. En caso de encontrarlo, construye la estructura en memoria y la devuelve al finalizar.

El formato buscado es el siguiente: \begin{enumerate}
\item El primer número leído indica la cantidad de filas de la matriz ({\tt nrow}) \item El segundo número leído indica la cantidad de columnas de la matriz ({\tt ncol}) \item El tercer número leído indica la cantidad de entradas no cero de la matriz ({\tt nnz}) \item Luego trata de leer {\tt nnz} números enteros, que representan los índices de fila donde hay elementos no cero ({\tt rx}) \item Luego trata de leer {\tt nnz} números enteros, que representan los índices de columna donde hay elementos no cero ({\tt cx}) \item Luego trata de leer {\tt nnz} números reales, que representan los índices las entradas no cero de la matriz ({\tt val}) \end{enumerate}
Todos los números pueden estar separados por cualquier cantidad de espacios, tabulaciones o fines de línea.

\begin{Desc}
\item[Nota:]La función \hyperlink{sp__coord_8c_65d8ffec5eb8263159ca9e6310c6e762}{save\_\-coord()} genera el formato esperado por esta función. \end{Desc}


Definición en la línea 84 del archivo sp\_\-coord.c.

Hace referencia a BAL\_\-ERROR, sp\_\-coord::cx, free\_\-coord(), sp\_\-coord::ncol, sp\_\-coord::nnz, sp\_\-coord::nrow, sp\_\-coord::rx, y sp\_\-coord::val.

Referenciado por bal\_\-load\_\-coord().

\begin{Code}\begin{verbatim}85 {
86     int x, i;
87     float y;
88     sp_coord *m;
89 
90     m = (sp_coord*)malloc(sizeof(sp_coord));
91 
92     if (fscanf(fp, " %d", &x) != 1) {
93         BAL_ERROR("No se pudo leer la cantidad de filas");
94         free(m);
95         return NULL;
96     }
97     m->nrow = x;
98 
99     if (fscanf(fp, " %d", &x) != 1) {
100         BAL_ERROR("No se pudo leer la cantidad de columnas");
101         free(m);
102         return NULL;
103     }
104     m->ncol = x;
105 
106     if (fscanf(fp, " %d", &x) != 1) {
107         BAL_ERROR("No se pudo leer la cantidad de elementos no cero");
108         free(m);
109         return NULL;
110     }
111     m->nnz = x;
112 
113     m->rx = (unsigned int*)malloc(sizeof(unsigned int) * x);
114     m->cx = (unsigned int*)malloc(sizeof(unsigned int) * x);
115     m->val = (double*)malloc(sizeof(double) * x);
116 
117     /* Lee los índices de fila */
118     for (i=0; i < m->nnz; ++i) {
119         if (fscanf(fp, " %d", &x) != 1) {
120             BAL_ERROR("No se pudo leer uno de los índices de fila");
121             free_coord(m);
122             return NULL;
123         }
124         m->rx[i] = (unsigned int)x;
125     }
126 
127     /* Lee los índices de columna */
128     for (i=0; i < m->nnz; ++i) {
129         if (fscanf(fp, " %d", &x) != 1) {
130             BAL_ERROR("No se pudo leer uno de los índices de columna");
131             free_coord(m);
132             return NULL;
133         }
134         m->cx[i] = (unsigned int)x;
135     }
136 
137     /* Lee las entradas de la matriz */
138     for (i=0; i < m->nnz; ++i) {
139         if (fscanf(fp, " %f", &y) != 1) {
140             BAL_ERROR("No se pudo leer una de las entradas de la matriz");
141             free_coord(m);
142             return NULL;
143         }
144         m->val[i] = (double)y;
145     }
146 
147     return m;
148 }
\end{verbatim}
\end{Code}


\hypertarget{sp__coord_8c_627349c38a6e7dc3792abd416651fb83}{
\index{sp\_\-coord.c@{sp\_\-coord.c}!mat2coord@{mat2coord}}
\index{mat2coord@{mat2coord}!sp_coord.c@{sp\_\-coord.c}}
\paragraph{\setlength{\rightskip}{0pt plus 5cm}{\bf sp\_\-coord} $\ast$ mat2coord (int {\em n}, \/  int {\em m}, \/  double $\ast$$\ast$ {\em mat})}\hfill}
\label{sp__coord_8c_627349c38a6e7dc3792abd416651fb83}


Genera la matriz dispersa equivalente a la matriz completa mat nxm. 

\begin{Desc}
\item[Parámetros:]
\begin{description}
\item[{\em n}]Cantidad total de filas en mat \item[{\em m}]Cantidad total de columnas en mat \item[{\em mat}]Matriz completa en el formato clásico de C. \end{description}
\end{Desc}


Definición en la línea 19 del archivo sp\_\-coord.c.

Hace referencia a sp\_\-coord::cx, sp\_\-coord::ncol, sp\_\-coord::nnz, sp\_\-coord::nrow, sp\_\-coord::rx, y sp\_\-coord::val.

Referenciado por bal\_\-mat2coord().

\begin{Code}\begin{verbatim}20 {
21     unsigned int i, j, count;
22     sp_coord *sp;
23 
24     count = 0;
25     for(i=0; i < n; ++i) {
26         for(j=0; j < m; ++j) {
27             if (mat[i][j] != 0)
28                 ++count;
29         }
30     }
31 
32     sp = (sp_coord*)malloc(sizeof(sp_coord));
33     sp->nrow = n;
34     sp->ncol = m;
35     sp->nnz = count;
36     sp->rx = NULL;
37     sp->cx = NULL;
38     sp->val = NULL;
39 
40     if (count > 0) {
41         sp->rx = (unsigned int*)malloc(sizeof(unsigned int) * count);
42         sp->cx = (unsigned int*)malloc(sizeof(unsigned int) * count);
43         sp->val = (double*)malloc(sizeof(double) * count);
44 
45         count = 0;
46         for(i=0; i < n; ++i) {
47             for(j=0; j < m; ++j) {
48                 if (mat[i][j] != 0) {
49                     sp->rx[count] = i;
50                     sp->cx[count] = j;
51                     sp->val[count] = mat[i][j];
52                     ++count;
53                 }
54             }
55         }
56     }
57 
58     return sp;
59 }
\end{verbatim}
\end{Code}


\hypertarget{sp__coord_8c_65d8ffec5eb8263159ca9e6310c6e762}{
\index{sp\_\-coord.c@{sp\_\-coord.c}!save\_\-coord@{save\_\-coord}}
\index{save\_\-coord@{save\_\-coord}!sp_coord.c@{sp\_\-coord.c}}
\paragraph{\setlength{\rightskip}{0pt plus 5cm}void save\_\-coord (FILE $\ast$ {\em fp}, \/  {\bf sp\_\-coord} $\ast$ {\em A})}\hfill}
\label{sp__coord_8c_65d8ffec5eb8263159ca9e6310c6e762}


Escribe en {\tt fp} la matriz {\tt A} en un formato entendible por \hyperlink{sp__coord_8c_d6a1042d41ff43da1f26396f5803daac}{load\_\-coord()}. 

Esta función es útil para respaldar matrices en formato simple por coordenadas. Una matriz guardada mediante esta función puede ser cargada nuevamente mediante \hyperlink{sp__coord_8c_d6a1042d41ff43da1f26396f5803daac}{load\_\-coord()}. 

Definición en la línea 156 del archivo sp\_\-coord.c.

Hace referencia a sp\_\-coord::cx, sp\_\-coord::ncol, sp\_\-coord::nnz, sp\_\-coord::nrow, sp\_\-coord::rx, y sp\_\-coord::val.

Referenciado por bal\_\-save\_\-coord().

\begin{Code}\begin{verbatim}157 {
158     unsigned int i;
159 
160     fprintf(fp, "%d\n", A->nrow);
161     fprintf(fp, "%d\n", A->ncol);
162     fprintf(fp, "%d\n", A->nnz);
163 
164     for (i=0; i < A->nnz; ++i)
165         fprintf(fp, " %d", A->rx[i]);
166     fprintf(fp, "\n");
167 
168     for (i=0; i < A->nnz; ++i)
169         fprintf(fp, " %d", A->cx[i]);
170     fprintf(fp, "\n");
171 
172     for (i=0; i < A->nnz; ++i)
173         fprintf(fp, " %g", A->val[i]);
174     fprintf(fp, "\n");
175 }
\end{verbatim}
\end{Code}


\hypertarget{sp__coord_8c_ff646304bb38349c4620069c73a35760}{
\index{sp\_\-coord.c@{sp\_\-coord.c}!sp\_\-imprimir\_\-coord@{sp\_\-imprimir\_\-coord}}
\index{sp\_\-imprimir\_\-coord@{sp\_\-imprimir\_\-coord}!sp_coord.c@{sp\_\-coord.c}}
\paragraph{\setlength{\rightskip}{0pt plus 5cm}void sp\_\-imprimir\_\-coord (FILE $\ast$ {\em fp}, \/  {\bf sp\_\-coord} $\ast$ {\em mat})}\hfill}
\label{sp__coord_8c_ff646304bb38349c4620069c73a35760}


Imprime la matriz guardada en formato simple por coordenadas en fp. 

\begin{Desc}
\item[Parámetros:]
\begin{description}
\item[{\em fp}]Archivo en el cual se imprimirá la matriz \item[{\em mat}]Matriz a imprimir en formato simple por coordenadas \end{description}
\end{Desc}


Definición en la línea 183 del archivo sp\_\-coord.c.

Hace referencia a sp\_\-coord::cx, sp\_\-coord::ncol, sp\_\-coord::nnz, sp\_\-coord::nrow, sp\_\-coord::rx, y sp\_\-coord::val.

Referenciado por bal\_\-imprimir\_\-coord().

\begin{Code}\begin{verbatim}184 {
185     int i;
186 
187     if (mat == NULL) {
188         fprintf(fp, "Matriz nula\n");
189         return;
190     }
191 
192     fprintf(fp, "Cantidad de filas: %d\n", mat->nrow);
193     fprintf(fp, "Cantidad de columnas: %d\n", mat->ncol);
194     fprintf(fp, "Cantidad de elementos no cero: %d\n", mat->nnz);
195 
196     for(i=0; i < mat->nnz; ++i) {
197         fprintf(fp, "(%d) [%d,%d] = %g\n", i, mat->rx[i], mat->cx[i], mat->val[i]);
198     }
199 }
\end{verbatim}
\end{Code}


