\hypertarget{bal_8c}{
\subsection{Referencia del Archivo bal.c}
\label{bal_8c}\index{bal.c@{bal.c}}
}
Biblioteca de Algebra Lineal, archivo principal. 

{\tt \#include $<$glib.h$>$}\par
{\tt \#include $<$stdio.h$>$}\par
{\tt \#include \char`\"{}bal.h\char`\"{}}\par
{\tt \#include \char`\"{}oper.h\char`\"{}}\par
{\tt \#include \char`\"{}sparse/sp\_\-coord.h\char`\"{}}\par
{\tt \#include \char`\"{}sparse/sp\_\-packcol.h\char`\"{}}\par
{\tt \#include \char`\"{}sparse/sp\_\-cds.h\char`\"{}}\par
{\tt \#include \char`\"{}cholesky/cholesky.h\char`\"{}}\par
{\tt \#include \char`\"{}cholesky/reordenamiento.h\char`\"{}}\par
\subsubsection*{Funciones}
\begin{CompactItemize}
\item 
int \hyperlink{bal_8c_0611775fcb04cf55d38106686c13e33e}{yyparse} (const char $\ast$archivo, double $\ast$$\ast$$\ast$matriz, int $\ast$n, int $\ast$m)
\begin{CompactList}\small\item\em Parser de matrices en formato matlab generado con bison y flex. \item\end{CompactList}\item 
int \hyperlink{bal_8c_5fa9afcb6f46fadec6d601f11592ebc9}{bal\_\-cargar\_\-matriz} (const char $\ast$archivo, double $\ast$$\ast$$\ast$matriz, int $\ast$n, int $\ast$m)
\begin{CompactList}\small\item\em Carga una matriz en memoria desde un archivo. \item\end{CompactList}\item 
\hypertarget{bal_8c_001bc810dfe24d78c6ee465351471ca7}{
void \hyperlink{bal_8c_001bc810dfe24d78c6ee465351471ca7}{bal\_\-imprimir\_\-matriz} (FILE $\ast$fp, double $\ast$$\ast$matriz, int n, int m)}
\label{bal_8c_001bc810dfe24d78c6ee465351471ca7}

\begin{CompactList}\small\item\em Imprime la matriz en el archivo {\tt fp}. \item\end{CompactList}\item 
\hyperlink{structsp__coord}{sp\_\-coord} $\ast$ \hyperlink{bal_8c_03258fb282226ce95701827d775bb011}{bal\_\-mat2coord} (int n, int m, double $\ast$$\ast$mat)
\begin{CompactList}\small\item\em Genera la matriz dispersa equivalente a la matriz completa mat (n x m). \item\end{CompactList}\item 
\hyperlink{structsp__packcol}{sp\_\-packcol} $\ast$ \hyperlink{bal_8c_6ced93fa5fb8f0a5103d8ee2e2bb58c2}{bal\_\-coord2packcol} (\hyperlink{structsp__coord}{sp\_\-coord} $\ast$mat)
\begin{CompactList}\small\item\em Genera una instancia de la matriz mat en formato empaquetado por columna. \item\end{CompactList}\item 
\hyperlink{structsp__packcol}{sp\_\-packcol} $\ast$ \hyperlink{bal_8c_069fc47d55dc12786bae9f18382344e5}{bal\_\-coord2packcol\_\-symmetric} (\hyperlink{structsp__coord}{sp\_\-coord} $\ast$mat)
\begin{CompactList}\small\item\em Genera una instancia de la matriz mat en formato empaquetado por columna para matrices simétricas. \item\end{CompactList}\item 
void \hyperlink{bal_8c_fb2e0eb90b54ac0b083d42f1d38d45ac}{bal\_\-imprimir\_\-coord} (FILE $\ast$fp, \hyperlink{structsp__coord}{sp\_\-coord} $\ast$mat)
\begin{CompactList}\small\item\em Imprime la matriz guardada en formato simple por coordenadas en fp. \item\end{CompactList}\item 
void \hyperlink{bal_8c_5c7bdb5d92a964b7e4cc66d8058e6fed}{bal\_\-imprimir\_\-packcol} (FILE $\ast$fp, \hyperlink{structsp__packcol}{sp\_\-packcol} $\ast$mat)
\begin{CompactList}\small\item\em Imprime la matriz guardada en formato empaquetado por columna en fp. \item\end{CompactList}\item 
void \hyperlink{bal_8c_dfee3769ca35f010759eddbf8e9cc476}{bal\_\-mult\_\-vec\_\-packcol\_\-symmetric} (\hyperlink{structsp__packcol}{sp\_\-packcol} $\ast$A, double $\ast$x, double $\ast$y)
\begin{CompactList}\small\item\em Multiplica una matriz simétrica empaquetada por columna por un vector. \item\end{CompactList}\item 
void \hyperlink{bal_8c_413951b2a155feea5c67288a3ccb34bf}{bal\_\-mult\_\-vec\_\-packcol} (\hyperlink{structsp__packcol}{sp\_\-packcol} $\ast$A, double $\ast$x, double $\ast$y)
\begin{CompactList}\small\item\em Multiplica una matriz empaquetada por columna por un vector. \item\end{CompactList}\item 
int \hyperlink{bal_8c_27f481dd5d13dfab5b06bb95d5c5aa1b}{bal\_\-row\_\-traversal\_\-packcol} (\hyperlink{structsp__packcol}{sp\_\-packcol} $\ast$A, int $\ast$i, int $\ast$j, int $\ast$posij)
\begin{CompactList}\small\item\em Implementa un mecanismo eficiente para recorrer por filas una matriz dispersa empaquetada por columnas. \item\end{CompactList}\item 
\hyperlink{structsp__coord}{sp\_\-coord} $\ast$ \hyperlink{bal_8c_1e8a293081b94602060a2f757e92a8ce}{bal\_\-load\_\-coord} (FILE $\ast$fp)
\begin{CompactList}\small\item\em Escribe en {\tt fp} la matriz {\tt A} en un formato entendible por \hyperlink{sp__coord_8c_d6a1042d41ff43da1f26396f5803daac}{load\_\-coord()}. \item\end{CompactList}\item 
void \hyperlink{bal_8c_abba0c987c1cd21f99ae3ef4a5fca7ac}{bal\_\-save\_\-coord} (FILE $\ast$fp, \hyperlink{structsp__coord}{sp\_\-coord} $\ast$A)
\begin{CompactList}\small\item\em Escribe en {\tt fp} la matriz {\tt A} en un formato entendible por \hyperlink{sp__coord_8c_d6a1042d41ff43da1f26396f5803daac}{load\_\-coord()}. \item\end{CompactList}\item 
void \hyperlink{bal_8c_5ae3c23f89da32858f1420e32d35ccdc}{bal\_\-save\_\-packcol\_\-symmetric} (FILE $\ast$fp, \hyperlink{structsp__packcol}{sp\_\-packcol} $\ast$A)
\begin{CompactList}\small\item\em Imprime la matriz simétrica empaquetada por columna en formato matlab en el archivo fp. \item\end{CompactList}\item 
void \hyperlink{bal_8c_23df03ad23589aefe5103d463d55ff8c}{bal\_\-save\_\-packcol} (FILE $\ast$fp, \hyperlink{structsp__packcol}{sp\_\-packcol} $\ast$A)
\begin{CompactList}\small\item\em Imprime la matriz {\tt A} en formato matlab en el archivo {\tt fp}. \item\end{CompactList}\item 
int $\ast$ \hyperlink{bal_8c_e282c362416526cbcedf0c9c49f8a8d7}{bal\_\-elimination\_\-tree} (\hyperlink{structsp__packcol}{sp\_\-packcol} $\ast$A, int $\ast$nnz)
\begin{CompactList}\small\item\em Calcula el árbol de eliminación de la matriz simétrica A. \item\end{CompactList}\item 
\hyperlink{structsp__packcol}{sp\_\-packcol} $\ast$ \hyperlink{bal_8c_5423521d4378c7a23082256da6b7eba9}{bal\_\-symbolic\_\-factorization} (\hyperlink{structsp__packcol}{sp\_\-packcol} $\ast$A)
\begin{CompactList}\small\item\em Factorización simbólica de la matriz A. \item\end{CompactList}\item 
void \hyperlink{bal_8c_9bfee73c39b529b912b2f140e0ff1765}{bal\_\-numerical\_\-factorization} (\hyperlink{structsp__packcol}{sp\_\-packcol} $\ast$A, \hyperlink{structsp__packcol}{sp\_\-packcol} $\ast$L)
\begin{CompactList}\small\item\em Sobreescribe {\tt L} con la factorización de Cholesky de A. \item\end{CompactList}\item 
void \hyperlink{bal_8c_1bb603f980bf7cd3476b13a2f61e9a87}{bal\_\-cholesky\_\-Lsolver} (\hyperlink{structsp__packcol}{sp\_\-packcol} $\ast$L, double $\ast$b)
\begin{CompactList}\small\item\em Sobrescribe {\tt b} con la solución del sistema $Ly=b$. \item\end{CompactList}\item 
void \hyperlink{bal_8c_00c65a19e4490139c1abe23a72a19c04}{bal\_\-cholesky\_\-LTsolver} (\hyperlink{structsp__packcol}{sp\_\-packcol} $\ast$L, double $\ast$b)
\begin{CompactList}\small\item\em Sobreescribe {\tt b} con la solución de $L^Tx=b$. \item\end{CompactList}\item 
void \hyperlink{bal_8c_c27eba96470b3ac5a5019230bd380a1a}{bal\_\-cholesky\_\-solver} (\hyperlink{structsp__packcol}{sp\_\-packcol} $\ast$A, double $\ast$b)
\begin{CompactList}\small\item\em Resuelve un sistema lineal mediante el algoritmo de Cholesky optimizado para matrices dispersas. \item\end{CompactList}\item 
\hyperlink{structsp__cds}{sp\_\-cds} $\ast$ \hyperlink{bal_8c_26024ae0fd4f685b495f1e33a6e65da8}{bal\_\-coord2cds} (\hyperlink{structsp__coord}{sp\_\-coord} $\ast$mat)
\begin{CompactList}\small\item\em Genera la matriz en formato CDS equivalente a la matriz mat en formato simple. \item\end{CompactList}\item 
void \hyperlink{bal_8c_0098b5496a5f2131a34c197bbc9ba57a}{bal\_\-imprimir\_\-cds} (FILE $\ast$fp, \hyperlink{structsp__cds}{sp\_\-cds} $\ast$mat)
\begin{CompactList}\small\item\em Imprime la matriz guardada en formato simple por coordenadas en fp. \item\end{CompactList}\item 
void \hyperlink{bal_8c_73ce65d5fd284423c38517d6370f449e}{bal\_\-mult\_\-vec\_\-cds} (\hyperlink{structsp__cds}{sp\_\-cds} $\ast$A, double $\ast$x, double $\ast$y)
\begin{CompactList}\small\item\em Multiplica una matriz en formato CDS por un vector. \item\end{CompactList}\item 
\hyperlink{structsp__cds}{sp\_\-cds} $\ast$ \hyperlink{bal_8c_1500ac2a006e9229a2a4fe810734367b}{bal\_\-mult\_\-mat\_\-cds} (\hyperlink{structsp__cds}{sp\_\-cds} $\ast$A, \hyperlink{structsp__cds}{sp\_\-cds} $\ast$B)
\begin{CompactList}\small\item\em Multiplica dos matrices en formato CDS. \item\end{CompactList}\item 
void \hyperlink{bal_8c_5928b3ea635bf59c0a99e5c835c06668}{bal\_\-save\_\-cds} (FILE $\ast$fp, \hyperlink{structsp__cds}{sp\_\-cds} $\ast$A)
\begin{CompactList}\small\item\em Imprime la matriz {\tt A} en formato matlab en el archivo {\tt fp}. \item\end{CompactList}\item 
\hyperlink{structsp__packcol}{sp\_\-packcol} $\ast$ \hyperlink{bal_8c_cf73a00379402d81de059fe1af3c496a}{bal\_\-permutar\_\-packcol} (unsigned int $\ast$p, \hyperlink{structsp__packcol}{sp\_\-packcol} $\ast$A)
\begin{CompactList}\small\item\em Aplica una permutación de columnas sobre la matriz {\tt A}. \item\end{CompactList}\item 
void \hyperlink{bal_8c_b3ce8c7df5d3d2a8eb9cd70534f628ed}{bal\_\-free\_\-coord} (\hyperlink{structsp__coord}{sp\_\-coord} $\ast$A)
\begin{CompactList}\small\item\em Libera la memoria reservada por la matriz {\tt A}. \item\end{CompactList}\item 
void \hyperlink{bal_8c_ecd6ad512d46f1825e79eda4647eaf60}{bal\_\-free\_\-packcol} (\hyperlink{structsp__packcol}{sp\_\-packcol} $\ast$A)
\begin{CompactList}\small\item\em Libera la memoria reservada por la matriz {\tt A}. \item\end{CompactList}\item 
void \hyperlink{bal_8c_7eae9c8721058a53a00d0c0dad2551f2}{bal\_\-free\_\-cds} (\hyperlink{structsp__cds}{sp\_\-cds} $\ast$A)
\begin{CompactList}\small\item\em Libera la memoria reservada por la matriz {\tt A}. \item\end{CompactList}\end{CompactItemize}


\subsubsection{Descripción detallada}
Biblioteca de Algebra Lineal, archivo principal. 

Este archivo implementa las funciones disponibles mediante BAL. 

Definición en el archivo \hyperlink{bal_8c-source}{bal.c}.

\subsubsection{Documentación de las funciones}
\hypertarget{bal_8c_5fa9afcb6f46fadec6d601f11592ebc9}{
\index{bal.c@{bal.c}!bal\_\-cargar\_\-matriz@{bal\_\-cargar\_\-matriz}}
\index{bal\_\-cargar\_\-matriz@{bal\_\-cargar\_\-matriz}!bal.c@{bal.c}}
\paragraph{\setlength{\rightskip}{0pt plus 5cm}int bal\_\-cargar\_\-matriz (const char $\ast$ {\em archivo}, \/  double $\ast$$\ast$$\ast$ {\em matriz}, \/  int $\ast$ {\em n}, \/  int $\ast$ {\em m})}\hfill}
\label{bal_8c_5fa9afcb6f46fadec6d601f11592ebc9}


Carga una matriz en memoria desde un archivo. 

\begin{Desc}
\item[Parámetros:]
\begin{description}
\item[{\em archivo}]ENTRADA: El camino al archivo en el que esta definida la matriz \item[{\em matriz}]SALIDA: Puntero a la estructura de memoria donde fue alocada la matriz \item[{\em n}]SALIDA: Cantidad de filas \item[{\em m}]SALIDA: Cantidad de columnas\end{description}
\end{Desc}
La matriz es cargada en memoria de forma \char`\"{}convencional\char`\"{}. El formato esperado es el mismo formato usado por matlab para definir matrices. El parser fue implementado utilizando las herramientas bison y flex de forma combinada. La función retorna 0 si todo funciono correctamente y otro valor en caso contrario. 

Definición en la línea 43 del archivo bal.c.

Hace referencia a yyparse().

\begin{Code}\begin{verbatim}44 {
45     if (yyparse(archivo, matriz, n, m) != 0)
46         return -1;
47 
48     return 0;
49 }
\end{verbatim}
\end{Code}


\hypertarget{bal_8c_1bb603f980bf7cd3476b13a2f61e9a87}{
\index{bal.c@{bal.c}!bal\_\-cholesky\_\-Lsolver@{bal\_\-cholesky\_\-Lsolver}}
\index{bal\_\-cholesky\_\-Lsolver@{bal\_\-cholesky\_\-Lsolver}!bal.c@{bal.c}}
\paragraph{\setlength{\rightskip}{0pt plus 5cm}void bal\_\-cholesky\_\-Lsolver ({\bf sp\_\-packcol} $\ast$ {\em L}, \/  double $\ast$ {\em b})}\hfill}
\label{bal_8c_1bb603f980bf7cd3476b13a2f61e9a87}


Sobrescribe {\tt b} con la solución del sistema $Ly=b$. 

Por más información vea \hyperlink{cholesky_8c_700b53e08123405fc54742fb52098588}{cholesky\_\-Lsolver()}. 

Definición en la línea 222 del archivo bal.c.

Hace referencia a cholesky\_\-Lsolver().

\begin{Code}\begin{verbatim}223 {
224     cholesky_Lsolver(L, b);
225 }
\end{verbatim}
\end{Code}


\hypertarget{bal_8c_00c65a19e4490139c1abe23a72a19c04}{
\index{bal.c@{bal.c}!bal\_\-cholesky\_\-LTsolver@{bal\_\-cholesky\_\-LTsolver}}
\index{bal\_\-cholesky\_\-LTsolver@{bal\_\-cholesky\_\-LTsolver}!bal.c@{bal.c}}
\paragraph{\setlength{\rightskip}{0pt plus 5cm}void bal\_\-cholesky\_\-LTsolver ({\bf sp\_\-packcol} $\ast$ {\em L}, \/  double $\ast$ {\em b})}\hfill}
\label{bal_8c_00c65a19e4490139c1abe23a72a19c04}


Sobreescribe {\tt b} con la solución de $L^Tx=b$. 

Por más información vea \hyperlink{cholesky_8c_afba46e6cb519f5b021129cf06ece41e}{cholesky\_\-LTsolver()}. 

Definición en la línea 232 del archivo bal.c.

Hace referencia a cholesky\_\-LTsolver().

\begin{Code}\begin{verbatim}233 {
234     cholesky_LTsolver(L, b);
235 }
\end{verbatim}
\end{Code}


\hypertarget{bal_8c_c27eba96470b3ac5a5019230bd380a1a}{
\index{bal.c@{bal.c}!bal\_\-cholesky\_\-solver@{bal\_\-cholesky\_\-solver}}
\index{bal\_\-cholesky\_\-solver@{bal\_\-cholesky\_\-solver}!bal.c@{bal.c}}
\paragraph{\setlength{\rightskip}{0pt plus 5cm}void bal\_\-cholesky\_\-solver ({\bf sp\_\-packcol} $\ast$ {\em A}, \/  double $\ast$ {\em b})}\hfill}
\label{bal_8c_c27eba96470b3ac5a5019230bd380a1a}


Resuelve un sistema lineal mediante el algoritmo de Cholesky optimizado para matrices dispersas. 

Por más información vea \hyperlink{cholesky_8c_b61e2da86cddb63b2ef5d38b46a1d4ea}{cholesky\_\-solver()}. 

Definición en la línea 242 del archivo bal.c.

Hace referencia a cholesky\_\-solver().

\begin{Code}\begin{verbatim}243 {
244     cholesky_solver(A, b);
245 }
\end{verbatim}
\end{Code}


\hypertarget{bal_8c_26024ae0fd4f685b495f1e33a6e65da8}{
\index{bal.c@{bal.c}!bal\_\-coord2cds@{bal\_\-coord2cds}}
\index{bal\_\-coord2cds@{bal\_\-coord2cds}!bal.c@{bal.c}}
\paragraph{\setlength{\rightskip}{0pt plus 5cm}{\bf sp\_\-cds} $\ast$ bal\_\-coord2cds ({\bf sp\_\-coord} $\ast$ {\em mat})}\hfill}
\label{bal_8c_26024ae0fd4f685b495f1e33a6e65da8}


Genera la matriz en formato CDS equivalente a la matriz mat en formato simple. 

Por más información vea \hyperlink{sp__cds_8c_00ca63cd94cc53957997b1c6fe79b721}{coord2cds()}. 

Definición en la línea 252 del archivo bal.c.

Hace referencia a coord2cds().

\begin{Code}\begin{verbatim}253 {
254     return coord2cds(mat);
255 }
\end{verbatim}
\end{Code}


\hypertarget{bal_8c_6ced93fa5fb8f0a5103d8ee2e2bb58c2}{
\index{bal.c@{bal.c}!bal\_\-coord2packcol@{bal\_\-coord2packcol}}
\index{bal\_\-coord2packcol@{bal\_\-coord2packcol}!bal.c@{bal.c}}
\paragraph{\setlength{\rightskip}{0pt plus 5cm}{\bf sp\_\-packcol} $\ast$ bal\_\-coord2packcol ({\bf sp\_\-coord} $\ast$ {\em mat})}\hfill}
\label{bal_8c_6ced93fa5fb8f0a5103d8ee2e2bb58c2}


Genera una instancia de la matriz mat en formato empaquetado por columna. 

Por más información vea \hyperlink{sp__packcol_8c_729d10ec6867249f9b58ceb2683a3318}{coord2packcol()}. 

Definición en la línea 82 del archivo bal.c.

Hace referencia a coord2packcol().

\begin{Code}\begin{verbatim}83 {
84     return coord2packcol(mat);
85 }
\end{verbatim}
\end{Code}


\hypertarget{bal_8c_069fc47d55dc12786bae9f18382344e5}{
\index{bal.c@{bal.c}!bal\_\-coord2packcol\_\-symmetric@{bal\_\-coord2packcol\_\-symmetric}}
\index{bal\_\-coord2packcol\_\-symmetric@{bal\_\-coord2packcol\_\-symmetric}!bal.c@{bal.c}}
\paragraph{\setlength{\rightskip}{0pt plus 5cm}{\bf sp\_\-packcol} $\ast$ bal\_\-coord2packcol\_\-symmetric ({\bf sp\_\-coord} $\ast$ {\em mat})}\hfill}
\label{bal_8c_069fc47d55dc12786bae9f18382344e5}


Genera una instancia de la matriz mat en formato empaquetado por columna para matrices simétricas. 

Por más información vea \hyperlink{sp__packcol_8c_7d8ea564f906cc721e5de1c4a4f5844b}{coord2packcol\_\-symmetric()}. 

Definición en la línea 92 del archivo bal.c.

Hace referencia a coord2packcol\_\-symmetric().

\begin{Code}\begin{verbatim}93 {
94     return coord2packcol_symmetric(mat);
95 }
\end{verbatim}
\end{Code}


\hypertarget{bal_8c_e282c362416526cbcedf0c9c49f8a8d7}{
\index{bal.c@{bal.c}!bal\_\-elimination\_\-tree@{bal\_\-elimination\_\-tree}}
\index{bal\_\-elimination\_\-tree@{bal\_\-elimination\_\-tree}!bal.c@{bal.c}}
\paragraph{\setlength{\rightskip}{0pt plus 5cm}int $\ast$ bal\_\-elimination\_\-tree ({\bf sp\_\-packcol} $\ast$ {\em A}, \/  int $\ast$ {\em nnz})}\hfill}
\label{bal_8c_e282c362416526cbcedf0c9c49f8a8d7}


Calcula el árbol de eliminación de la matriz simétrica A. 

Por más información vea \hyperlink{cholesky_8c_c247ab532a0c89efa5fc80ccd1de6315}{elimination\_\-tree()}. 

Definición en la línea 192 del archivo bal.c.

Hace referencia a elimination\_\-tree().

\begin{Code}\begin{verbatim}193 {
194     return elimination_tree(A, nnz);
195 }
\end{verbatim}
\end{Code}


\hypertarget{bal_8c_7eae9c8721058a53a00d0c0dad2551f2}{
\index{bal.c@{bal.c}!bal\_\-free\_\-cds@{bal\_\-free\_\-cds}}
\index{bal\_\-free\_\-cds@{bal\_\-free\_\-cds}!bal.c@{bal.c}}
\paragraph{\setlength{\rightskip}{0pt plus 5cm}void bal\_\-free\_\-cds ({\bf sp\_\-cds} $\ast$ {\em A})}\hfill}
\label{bal_8c_7eae9c8721058a53a00d0c0dad2551f2}


Libera la memoria reservada por la matriz {\tt A}. 

Por más información ver \hyperlink{sp__cds_8c_6b5d5549f99d990b9a259d982e3e4485}{free\_\-cds()} 

Definición en la línea 332 del archivo bal.c.

Hace referencia a free\_\-cds().

\begin{Code}\begin{verbatim}333 {
334     free_cds(A);
335 }
\end{verbatim}
\end{Code}


\hypertarget{bal_8c_b3ce8c7df5d3d2a8eb9cd70534f628ed}{
\index{bal.c@{bal.c}!bal\_\-free\_\-coord@{bal\_\-free\_\-coord}}
\index{bal\_\-free\_\-coord@{bal\_\-free\_\-coord}!bal.c@{bal.c}}
\paragraph{\setlength{\rightskip}{0pt plus 5cm}void bal\_\-free\_\-coord ({\bf sp\_\-coord} $\ast$ {\em A})}\hfill}
\label{bal_8c_b3ce8c7df5d3d2a8eb9cd70534f628ed}


Libera la memoria reservada por la matriz {\tt A}. 

Por más información ver \hyperlink{sp__coord_8c_33dd472b98444c5d08c9099556226816}{free\_\-coord()} 

Definición en la línea 312 del archivo bal.c.

Hace referencia a free\_\-coord().

\begin{Code}\begin{verbatim}313 {
314     free_coord(A);
315 }
\end{verbatim}
\end{Code}


\hypertarget{bal_8c_ecd6ad512d46f1825e79eda4647eaf60}{
\index{bal.c@{bal.c}!bal\_\-free\_\-packcol@{bal\_\-free\_\-packcol}}
\index{bal\_\-free\_\-packcol@{bal\_\-free\_\-packcol}!bal.c@{bal.c}}
\paragraph{\setlength{\rightskip}{0pt plus 5cm}void bal\_\-free\_\-packcol ({\bf sp\_\-packcol} $\ast$ {\em A})}\hfill}
\label{bal_8c_ecd6ad512d46f1825e79eda4647eaf60}


Libera la memoria reservada por la matriz {\tt A}. 

Por más información ver \hyperlink{sp__packcol_8c_b9f6d408419504b0e7154b221bc0ca4b}{free\_\-packcol()} 

Definición en la línea 322 del archivo bal.c.

Hace referencia a free\_\-packcol().

\begin{Code}\begin{verbatim}323 {
324     free_packcol(A);
325 }
\end{verbatim}
\end{Code}


\hypertarget{bal_8c_0098b5496a5f2131a34c197bbc9ba57a}{
\index{bal.c@{bal.c}!bal\_\-imprimir\_\-cds@{bal\_\-imprimir\_\-cds}}
\index{bal\_\-imprimir\_\-cds@{bal\_\-imprimir\_\-cds}!bal.c@{bal.c}}
\paragraph{\setlength{\rightskip}{0pt plus 5cm}void bal\_\-imprimir\_\-cds (FILE $\ast$ {\em fp}, \/  {\bf sp\_\-cds} $\ast$ {\em mat})}\hfill}
\label{bal_8c_0098b5496a5f2131a34c197bbc9ba57a}


Imprime la matriz guardada en formato simple por coordenadas en fp. 

Por más información vea \hyperlink{sp__cds_8c_1f5bfb80b511e70892a29e07d4769e98}{sp\_\-imprimir\_\-cds()}. 

Definición en la línea 262 del archivo bal.c.

Hace referencia a sp\_\-imprimir\_\-cds().

\begin{Code}\begin{verbatim}263 {
264     sp_imprimir_cds(fp, mat);
265 }
\end{verbatim}
\end{Code}


\hypertarget{bal_8c_fb2e0eb90b54ac0b083d42f1d38d45ac}{
\index{bal.c@{bal.c}!bal\_\-imprimir\_\-coord@{bal\_\-imprimir\_\-coord}}
\index{bal\_\-imprimir\_\-coord@{bal\_\-imprimir\_\-coord}!bal.c@{bal.c}}
\paragraph{\setlength{\rightskip}{0pt plus 5cm}void bal\_\-imprimir\_\-coord (FILE $\ast$ {\em fp}, \/  {\bf sp\_\-coord} $\ast$ {\em mat})}\hfill}
\label{bal_8c_fb2e0eb90b54ac0b083d42f1d38d45ac}


Imprime la matriz guardada en formato simple por coordenadas en fp. 

Por más información vea \hyperlink{sp__coord_8c_ff646304bb38349c4620069c73a35760}{sp\_\-imprimir\_\-coord()}. 

Definición en la línea 102 del archivo bal.c.

Hace referencia a sp\_\-imprimir\_\-coord().

\begin{Code}\begin{verbatim}103 {
104     sp_imprimir_coord(fp, mat);
105 }
\end{verbatim}
\end{Code}


\hypertarget{bal_8c_5c7bdb5d92a964b7e4cc66d8058e6fed}{
\index{bal.c@{bal.c}!bal\_\-imprimir\_\-packcol@{bal\_\-imprimir\_\-packcol}}
\index{bal\_\-imprimir\_\-packcol@{bal\_\-imprimir\_\-packcol}!bal.c@{bal.c}}
\paragraph{\setlength{\rightskip}{0pt plus 5cm}void bal\_\-imprimir\_\-packcol (FILE $\ast$ {\em fp}, \/  {\bf sp\_\-packcol} $\ast$ {\em mat})}\hfill}
\label{bal_8c_5c7bdb5d92a964b7e4cc66d8058e6fed}


Imprime la matriz guardada en formato empaquetado por columna en fp. 

Por más información vea \hyperlink{sp__packcol_8c_b4d4d490381b43a9628f2001bb6c99d6}{sp\_\-imprimir\_\-packcol()}. 

Definición en la línea 112 del archivo bal.c.

Hace referencia a sp\_\-imprimir\_\-packcol().

\begin{Code}\begin{verbatim}113 {
114     sp_imprimir_packcol(fp, mat);
115 }
\end{verbatim}
\end{Code}


\hypertarget{bal_8c_1e8a293081b94602060a2f757e92a8ce}{
\index{bal.c@{bal.c}!bal\_\-load\_\-coord@{bal\_\-load\_\-coord}}
\index{bal\_\-load\_\-coord@{bal\_\-load\_\-coord}!bal.c@{bal.c}}
\paragraph{\setlength{\rightskip}{0pt plus 5cm}{\bf sp\_\-coord} $\ast$ bal\_\-load\_\-coord (FILE $\ast$ {\em fp})}\hfill}
\label{bal_8c_1e8a293081b94602060a2f757e92a8ce}


Escribe en {\tt fp} la matriz {\tt A} en un formato entendible por \hyperlink{sp__coord_8c_d6a1042d41ff43da1f26396f5803daac}{load\_\-coord()}. 

Por más información vea \hyperlink{sp__coord_8c_d6a1042d41ff43da1f26396f5803daac}{load\_\-coord()}. 

Definición en la línea 152 del archivo bal.c.

Hace referencia a load\_\-coord().

\begin{Code}\begin{verbatim}153 {
154     return load_coord(fp);
155 }
\end{verbatim}
\end{Code}


\hypertarget{bal_8c_03258fb282226ce95701827d775bb011}{
\index{bal.c@{bal.c}!bal\_\-mat2coord@{bal\_\-mat2coord}}
\index{bal\_\-mat2coord@{bal\_\-mat2coord}!bal.c@{bal.c}}
\paragraph{\setlength{\rightskip}{0pt plus 5cm}{\bf sp\_\-coord} $\ast$ bal\_\-mat2coord (int {\em n}, \/  int {\em m}, \/  double $\ast$$\ast$ {\em mat})}\hfill}
\label{bal_8c_03258fb282226ce95701827d775bb011}


Genera la matriz dispersa equivalente a la matriz completa mat (n x m). 

Por más información, vea \hyperlink{sp__coord_8c_627349c38a6e7dc3792abd416651fb83}{mat2coord()}. 

Definición en la línea 72 del archivo bal.c.

Hace referencia a mat2coord().

\begin{Code}\begin{verbatim}73 {
74     return mat2coord(n, m, mat);
75 }
\end{verbatim}
\end{Code}


\hypertarget{bal_8c_1500ac2a006e9229a2a4fe810734367b}{
\index{bal.c@{bal.c}!bal\_\-mult\_\-mat\_\-cds@{bal\_\-mult\_\-mat\_\-cds}}
\index{bal\_\-mult\_\-mat\_\-cds@{bal\_\-mult\_\-mat\_\-cds}!bal.c@{bal.c}}
\paragraph{\setlength{\rightskip}{0pt plus 5cm}{\bf sp\_\-cds} $\ast$ bal\_\-mult\_\-mat\_\-cds ({\bf sp\_\-cds} $\ast$ {\em A}, \/  {\bf sp\_\-cds} $\ast$ {\em B})}\hfill}
\label{bal_8c_1500ac2a006e9229a2a4fe810734367b}


Multiplica dos matrices en formato CDS. 

Por más información ver \hyperlink{oper_8c_6ea0383a6eb9a67301225f5f2fee87b4}{mult\_\-mat\_\-cds()}. 

Definición en la línea 282 del archivo bal.c.

Hace referencia a mult\_\-mat\_\-cds().

\begin{Code}\begin{verbatim}283 {
284     return mult_mat_cds(A, B);
285 }
\end{verbatim}
\end{Code}


\hypertarget{bal_8c_73ce65d5fd284423c38517d6370f449e}{
\index{bal.c@{bal.c}!bal\_\-mult\_\-vec\_\-cds@{bal\_\-mult\_\-vec\_\-cds}}
\index{bal\_\-mult\_\-vec\_\-cds@{bal\_\-mult\_\-vec\_\-cds}!bal.c@{bal.c}}
\paragraph{\setlength{\rightskip}{0pt plus 5cm}void bal\_\-mult\_\-vec\_\-cds ({\bf sp\_\-cds} $\ast$ {\em A}, \/  double $\ast$ {\em x}, \/  double $\ast$ {\em y})}\hfill}
\label{bal_8c_73ce65d5fd284423c38517d6370f449e}


Multiplica una matriz en formato CDS por un vector. 

Por más información ver \hyperlink{oper_8c_695a940f46d1c589bfcf8758f7a87f15}{mult\_\-vec\_\-cds()}. 

Definición en la línea 272 del archivo bal.c.

Hace referencia a mult\_\-vec\_\-cds().

\begin{Code}\begin{verbatim}273 {
274     mult_vec_cds(A, x, y);
275 }
\end{verbatim}
\end{Code}


\hypertarget{bal_8c_413951b2a155feea5c67288a3ccb34bf}{
\index{bal.c@{bal.c}!bal\_\-mult\_\-vec\_\-packcol@{bal\_\-mult\_\-vec\_\-packcol}}
\index{bal\_\-mult\_\-vec\_\-packcol@{bal\_\-mult\_\-vec\_\-packcol}!bal.c@{bal.c}}
\paragraph{\setlength{\rightskip}{0pt plus 5cm}void bal\_\-mult\_\-vec\_\-packcol ({\bf sp\_\-packcol} $\ast$ {\em A}, \/  double $\ast$ {\em x}, \/  double $\ast$ {\em y})}\hfill}
\label{bal_8c_413951b2a155feea5c67288a3ccb34bf}


Multiplica una matriz empaquetada por columna por un vector. 

Por más información vea \hyperlink{oper_8c_8bdc34ab9889aa1cad64fcc44dd7a2ec}{mult\_\-vec\_\-packcol()}. 

Definición en la línea 132 del archivo bal.c.

Hace referencia a mult\_\-vec\_\-packcol().

\begin{Code}\begin{verbatim}133 {
134     mult_vec_packcol(A, x, y);
135 }
\end{verbatim}
\end{Code}


\hypertarget{bal_8c_dfee3769ca35f010759eddbf8e9cc476}{
\index{bal.c@{bal.c}!bal\_\-mult\_\-vec\_\-packcol\_\-symmetric@{bal\_\-mult\_\-vec\_\-packcol\_\-symmetric}}
\index{bal\_\-mult\_\-vec\_\-packcol\_\-symmetric@{bal\_\-mult\_\-vec\_\-packcol\_\-symmetric}!bal.c@{bal.c}}
\paragraph{\setlength{\rightskip}{0pt plus 5cm}void bal\_\-mult\_\-vec\_\-packcol\_\-symmetric ({\bf sp\_\-packcol} $\ast$ {\em A}, \/  double $\ast$ {\em x}, \/  double $\ast$ {\em y})}\hfill}
\label{bal_8c_dfee3769ca35f010759eddbf8e9cc476}


Multiplica una matriz simétrica empaquetada por columna por un vector. 

Por más información vea \hyperlink{oper_8c_0f9cb665556ffdd8bdb5dfbe861e0d28}{mult\_\-vec\_\-packcol\_\-symmetric()}. 

Definición en la línea 122 del archivo bal.c.

Hace referencia a mult\_\-vec\_\-packcol\_\-symmetric().

\begin{Code}\begin{verbatim}123 {
124     mult_vec_packcol_symmetric(A, x, y);
125 }
\end{verbatim}
\end{Code}


\hypertarget{bal_8c_9bfee73c39b529b912b2f140e0ff1765}{
\index{bal.c@{bal.c}!bal\_\-numerical\_\-factorization@{bal\_\-numerical\_\-factorization}}
\index{bal\_\-numerical\_\-factorization@{bal\_\-numerical\_\-factorization}!bal.c@{bal.c}}
\paragraph{\setlength{\rightskip}{0pt plus 5cm}void bal\_\-numerical\_\-factorization ({\bf sp\_\-packcol} $\ast$ {\em A}, \/  {\bf sp\_\-packcol} $\ast$ {\em L})}\hfill}
\label{bal_8c_9bfee73c39b529b912b2f140e0ff1765}


Sobreescribe {\tt L} con la factorización de Cholesky de A. 

Por más información vea \hyperlink{cholesky_8c_2dd1cb8f4a30d7950399b658862a2180}{numerical\_\-factorization()}. 

Definición en la línea 212 del archivo bal.c.

Hace referencia a numerical\_\-factorization().

\begin{Code}\begin{verbatim}213 {
214     numerical_factorization(A, L);
215 }
\end{verbatim}
\end{Code}


\hypertarget{bal_8c_cf73a00379402d81de059fe1af3c496a}{
\index{bal.c@{bal.c}!bal\_\-permutar\_\-packcol@{bal\_\-permutar\_\-packcol}}
\index{bal\_\-permutar\_\-packcol@{bal\_\-permutar\_\-packcol}!bal.c@{bal.c}}
\paragraph{\setlength{\rightskip}{0pt plus 5cm}{\bf sp\_\-packcol} $\ast$ bal\_\-permutar\_\-packcol (unsigned int $\ast$ {\em p}, \/  {\bf sp\_\-packcol} $\ast$ {\em A})}\hfill}
\label{bal_8c_cf73a00379402d81de059fe1af3c496a}


Aplica una permutación de columnas sobre la matriz {\tt A}. 

Por más información vea \hyperlink{reordenamiento_8c_05f0067f3214ea7f66d4519c09b5c5c2}{permutar\_\-packcol()}. 

Definición en la línea 302 del archivo bal.c.

Hace referencia a permutar\_\-packcol().

\begin{Code}\begin{verbatim}303 {
304     return permutar_packcol(p, A);
305 }
\end{verbatim}
\end{Code}


\hypertarget{bal_8c_27f481dd5d13dfab5b06bb95d5c5aa1b}{
\index{bal.c@{bal.c}!bal\_\-row\_\-traversal\_\-packcol@{bal\_\-row\_\-traversal\_\-packcol}}
\index{bal\_\-row\_\-traversal\_\-packcol@{bal\_\-row\_\-traversal\_\-packcol}!bal.c@{bal.c}}
\paragraph{\setlength{\rightskip}{0pt plus 5cm}int bal\_\-row\_\-traversal\_\-packcol ({\bf sp\_\-packcol} $\ast$ {\em A}, \/  int $\ast$ {\em i}, \/  int $\ast$ {\em j}, \/  int $\ast$ {\em posij})}\hfill}
\label{bal_8c_27f481dd5d13dfab5b06bb95d5c5aa1b}


Implementa un mecanismo eficiente para recorrer por filas una matriz dispersa empaquetada por columnas. 

Por más información vea \hyperlink{sp__packcol_8c_49b361aae383b0fd323f5735f0ff01bc}{row\_\-traversal\_\-packcol()}. 

Definición en la línea 142 del archivo bal.c.

Hace referencia a row\_\-traversal\_\-packcol().

\begin{Code}\begin{verbatim}143 {
144     return row_traversal_packcol(A, i, j, posij);
145 }
\end{verbatim}
\end{Code}


\hypertarget{bal_8c_5928b3ea635bf59c0a99e5c835c06668}{
\index{bal.c@{bal.c}!bal\_\-save\_\-cds@{bal\_\-save\_\-cds}}
\index{bal\_\-save\_\-cds@{bal\_\-save\_\-cds}!bal.c@{bal.c}}
\paragraph{\setlength{\rightskip}{0pt plus 5cm}void bal\_\-save\_\-cds (FILE $\ast$ {\em fp}, \/  {\bf sp\_\-cds} $\ast$ {\em A})}\hfill}
\label{bal_8c_5928b3ea635bf59c0a99e5c835c06668}


Imprime la matriz {\tt A} en formato matlab en el archivo {\tt fp}. 

Por más información vea \hyperlink{sp__cds_8c_dbc8b3571c8ace5419d2e02318988388}{save\_\-cds()}. 

Definición en la línea 292 del archivo bal.c.

Hace referencia a save\_\-cds().

\begin{Code}\begin{verbatim}293 {
294     save_cds(fp, A);
295 }
\end{verbatim}
\end{Code}


\hypertarget{bal_8c_abba0c987c1cd21f99ae3ef4a5fca7ac}{
\index{bal.c@{bal.c}!bal\_\-save\_\-coord@{bal\_\-save\_\-coord}}
\index{bal\_\-save\_\-coord@{bal\_\-save\_\-coord}!bal.c@{bal.c}}
\paragraph{\setlength{\rightskip}{0pt plus 5cm}void bal\_\-save\_\-coord (FILE $\ast$ {\em fp}, \/  {\bf sp\_\-coord} $\ast$ {\em A})}\hfill}
\label{bal_8c_abba0c987c1cd21f99ae3ef4a5fca7ac}


Escribe en {\tt fp} la matriz {\tt A} en un formato entendible por \hyperlink{sp__coord_8c_d6a1042d41ff43da1f26396f5803daac}{load\_\-coord()}. 

Por más información vea \hyperlink{sp__coord_8c_65d8ffec5eb8263159ca9e6310c6e762}{save\_\-coord()}. 

Definición en la línea 162 del archivo bal.c.

Hace referencia a save\_\-coord().

\begin{Code}\begin{verbatim}163 {
164     save_coord(fp, A);
165 }
\end{verbatim}
\end{Code}


\hypertarget{bal_8c_23df03ad23589aefe5103d463d55ff8c}{
\index{bal.c@{bal.c}!bal\_\-save\_\-packcol@{bal\_\-save\_\-packcol}}
\index{bal\_\-save\_\-packcol@{bal\_\-save\_\-packcol}!bal.c@{bal.c}}
\paragraph{\setlength{\rightskip}{0pt plus 5cm}void bal\_\-save\_\-packcol (FILE $\ast$ {\em fp}, \/  {\bf sp\_\-packcol} $\ast$ {\em A})}\hfill}
\label{bal_8c_23df03ad23589aefe5103d463d55ff8c}


Imprime la matriz {\tt A} en formato matlab en el archivo {\tt fp}. 

Por más información vea \hyperlink{sp__packcol_8c_45eba0df395fde4b428c76d4e688ed62}{save\_\-packcol()}. 

Definición en la línea 182 del archivo bal.c.

Hace referencia a save\_\-packcol().

\begin{Code}\begin{verbatim}183 {
184     save_packcol(fp, A);
185 }
\end{verbatim}
\end{Code}


\hypertarget{bal_8c_5ae3c23f89da32858f1420e32d35ccdc}{
\index{bal.c@{bal.c}!bal\_\-save\_\-packcol\_\-symmetric@{bal\_\-save\_\-packcol\_\-symmetric}}
\index{bal\_\-save\_\-packcol\_\-symmetric@{bal\_\-save\_\-packcol\_\-symmetric}!bal.c@{bal.c}}
\paragraph{\setlength{\rightskip}{0pt plus 5cm}void bal\_\-save\_\-packcol\_\-symmetric (FILE $\ast$ {\em fp}, \/  {\bf sp\_\-packcol} $\ast$ {\em A})}\hfill}
\label{bal_8c_5ae3c23f89da32858f1420e32d35ccdc}


Imprime la matriz simétrica empaquetada por columna en formato matlab en el archivo fp. 

Por más información vea \hyperlink{sp__packcol_8c_25a252f919f868ce004aa25b45d44ff3}{save\_\-packcol\_\-symmetric()}. 

Definición en la línea 172 del archivo bal.c.

Hace referencia a save\_\-packcol\_\-symmetric().

\begin{Code}\begin{verbatim}173 {
174     save_packcol_symmetric(fp, A);
175 }
\end{verbatim}
\end{Code}


\hypertarget{bal_8c_5423521d4378c7a23082256da6b7eba9}{
\index{bal.c@{bal.c}!bal\_\-symbolic\_\-factorization@{bal\_\-symbolic\_\-factorization}}
\index{bal\_\-symbolic\_\-factorization@{bal\_\-symbolic\_\-factorization}!bal.c@{bal.c}}
\paragraph{\setlength{\rightskip}{0pt plus 5cm}{\bf sp\_\-packcol} $\ast$ bal\_\-symbolic\_\-factorization ({\bf sp\_\-packcol} $\ast$ {\em A})}\hfill}
\label{bal_8c_5423521d4378c7a23082256da6b7eba9}


Factorización simbólica de la matriz A. 

Por más información vea \hyperlink{cholesky_8c_372ebd05f16cc5771aacf2d9e2497119}{symbolic\_\-factorization()}. 

Definición en la línea 202 del archivo bal.c.

Hace referencia a symbolic\_\-factorization().

\begin{Code}\begin{verbatim}203 {
204     return symbolic_factorization(A);
205 }
\end{verbatim}
\end{Code}


\hypertarget{bal_8c_0611775fcb04cf55d38106686c13e33e}{
\index{bal.c@{bal.c}!yyparse@{yyparse}}
\index{yyparse@{yyparse}!bal.c@{bal.c}}
\paragraph{\setlength{\rightskip}{0pt plus 5cm}int yyparse (const char $\ast$ {\em archivo}, \/  double $\ast$$\ast$$\ast$ {\em matriz}, \/  int $\ast$ {\em n}, \/  int $\ast$ {\em m})}\hfill}
\label{bal_8c_0611775fcb04cf55d38106686c13e33e}


Parser de matrices en formato matlab generado con bison y flex. 

\begin{Desc}
\item[Parámetros:]
\begin{description}
\item[{\em archivo}]ENTRADA: Camino al archivo donde esta definida la matriz \item[{\em matriz}]SALIDA: Matriz leída como lista de filas. Cada fila es una lista de valores. \item[{\em n}]SALIDA: Cantidad de filas de la matriz leída. \item[{\em m}]SALIDA: Cantidad de columnas de la matriz leída.\end{description}
\end{Desc}
{\bf NOTA:} Esta función es de uso interno de BAL. No debería ser necesario su uso externo. Vea la función bal\_\-cargar\_\-matriz. 

Referenciado por bal\_\-cargar\_\-matriz().