\hypertarget{sp__cds_8c}{
\subsection{Referencia del Archivo sparse/sp\_\-cds.c}
\label{sp__cds_8c}\index{sparse/sp\_\-cds.c@{sparse/sp\_\-cds.c}}
}
Archivo de implementación para matriz dispersa, formato simple. 

{\tt \#include $<$stdlib.h$>$}\par
{\tt \#include $<$stdio.h$>$}\par
{\tt \#include $<$glib.h$>$}\par
{\tt \#include \char`\"{}sp\_\-cds.h\char`\"{}}\par
{\tt \#include \char`\"{}../utils.h\char`\"{}}\par
\subsubsection*{Funciones}
\begin{CompactItemize}
\item 
\hyperlink{structsp__cds}{sp\_\-cds} $\ast$ \hyperlink{sp__cds_8c_00ca63cd94cc53957997b1c6fe79b721}{coord2cds} (\hyperlink{structsp__coord}{sp\_\-coord} $\ast$mat)
\begin{CompactList}\small\item\em Genera la matriz en formato CDS equivalente a la matriz mat en formato simple. \item\end{CompactList}\item 
void \hyperlink{sp__cds_8c_1f5bfb80b511e70892a29e07d4769e98}{sp\_\-imprimir\_\-cds} (FILE $\ast$fp, \hyperlink{structsp__cds}{sp\_\-cds} $\ast$mat)
\begin{CompactList}\small\item\em Imprime la matriz guardada en formato simple por coordenadas en fp. \item\end{CompactList}\item 
void \hyperlink{sp__cds_8c_dbc8b3571c8ace5419d2e02318988388}{save\_\-cds} (FILE $\ast$fp, \hyperlink{structsp__cds}{sp\_\-cds} $\ast$A)
\begin{CompactList}\small\item\em Imprime la matriz {\tt A} en formato matlab en el archivo {\tt fp}. \item\end{CompactList}\item 
void \hyperlink{sp__cds_8c_6b5d5549f99d990b9a259d982e3e4485}{free\_\-cds} (\hyperlink{structsp__cds}{sp\_\-cds} $\ast$A)
\begin{CompactList}\small\item\em Borra toda la memoria reservada por la matriz A. \item\end{CompactList}\end{CompactItemize}


\subsubsection{Descripción detallada}
Archivo de implementación para matriz dispersa, formato simple. 

Este archivo contiene la implementación de las funciones de utilidad para el formato de matriz dispersa simple. 

Definición en el archivo \hyperlink{sp__cds_8c-source}{sp\_\-cds.c}.

\subsubsection{Documentación de las funciones}
\hypertarget{sp__cds_8c_00ca63cd94cc53957997b1c6fe79b721}{
\index{sp\_\-cds.c@{sp\_\-cds.c}!coord2cds@{coord2cds}}
\index{coord2cds@{coord2cds}!sp_cds.c@{sp\_\-cds.c}}
\paragraph{\setlength{\rightskip}{0pt plus 5cm}{\bf sp\_\-cds} $\ast$ coord2cds ({\bf sp\_\-coord} $\ast$ {\em mat})}\hfill}
\label{sp__cds_8c_00ca63cd94cc53957997b1c6fe79b721}


Genera la matriz en formato CDS equivalente a la matriz mat en formato simple. 

\begin{Desc}
\item[Parámetros:]
\begin{description}
\item[{\em mat}]Matriz dispersa en formato simple. \end{description}
\end{Desc}
\begin{Desc}
\item[Devuelve:]Matriz equivalente en formato CDS. \end{Desc}


Definición en la línea 19 del archivo sp\_\-cds.c.

Hace referencia a binary\_\-search(), sp\_\-coord::cx, sp\_\-cds::dx, insert\_\-sorted(), sp\_\-cds::maxdiaglength, sp\_\-coord::ncol, sp\_\-cds::ncol, sp\_\-cds::ndiag, sp\_\-coord::nnz, sp\_\-coord::nrow, sp\_\-cds::nrow, sp\_\-coord::rx, sp\_\-coord::val, y sp\_\-cds::val.

Referenciado por bal\_\-coord2cds().

\begin{Code}\begin{verbatim}20 {
21     unsigned int n, m, tope, ii, i, j, dxl;
22     int diag, k;
23     sp_cds *cds;
24 
25     cds = (sp_cds*)malloc(sizeof(sp_cds));
26 
27     cds->nrow = n = mat->nrow;
28     cds->ncol = m = mat->ncol;
29     cds->maxdiaglength = MIN(n, m);
30     tope = m + n - 1; /* Cantidad máxima de diagonales */
31     cds->dx = (int*)malloc(sizeof(int) * tope);
32 
33     /* Busca diagonales con datos */
34     dxl = 0;
35     for (ii=0; ii < mat->nnz; ++ii) {
36         i = mat->rx[ii];
37         j = mat->cx[ii];
38         diag = j - i;
39 
40         if (binary_search(cds->dx, dxl, diag) == -1) {
41             insert_sorted(cds->dx, dxl - 1, diag);
42             ++dxl;
43         }
44     }
45 
46     /* Inicializa cds->val */
47     cds->ndiag = dxl;
48     cds->val = (double**)malloc(sizeof(double*) * dxl);
49     for (i=0; i < dxl; ++i) {
50         cds->val[i] = (double*)malloc(sizeof(double) * cds->maxdiaglength);
51         for (j=0; j < cds->maxdiaglength; ++j)
52             cds->val[i][j] = 0;
53     }
54 
55     /* Carga los valores */
56     for (ii=0; ii < mat->nnz; ++ii) {
57         i = mat->rx[ii];
58         j = mat->cx[ii];
59         diag = j - i;
60 
61         k = binary_search(cds->dx, dxl, diag);
62         cds->val[k][i] = mat->val[ii];
63     }
64 
65     return cds;
66 }
\end{verbatim}
\end{Code}


\hypertarget{sp__cds_8c_6b5d5549f99d990b9a259d982e3e4485}{
\index{sp\_\-cds.c@{sp\_\-cds.c}!free\_\-cds@{free\_\-cds}}
\index{free\_\-cds@{free\_\-cds}!sp_cds.c@{sp\_\-cds.c}}
\paragraph{\setlength{\rightskip}{0pt plus 5cm}void free\_\-cds ({\bf sp\_\-cds} $\ast$ {\em A})}\hfill}
\label{sp__cds_8c_6b5d5549f99d990b9a259d982e3e4485}


Borra toda la memoria reservada por la matriz A. 

Libera la memoria reservada por la estructura de datos \hyperlink{structsp__cds}{sp\_\-cds}. 

Definición en la línea 134 del archivo sp\_\-cds.c.

Hace referencia a sp\_\-cds::dx, sp\_\-cds::ndiag, y sp\_\-cds::val.

Referenciado por bal\_\-free\_\-cds().

\begin{Code}\begin{verbatim}135 {
136     unsigned int d;
137 
138     for (d=0; d < A->ndiag; ++d)
139         free(A->val[d]);
140 
141     free(A->val);
142     free(A->dx);
143     free(A);
144 }
\end{verbatim}
\end{Code}


\hypertarget{sp__cds_8c_dbc8b3571c8ace5419d2e02318988388}{
\index{sp\_\-cds.c@{sp\_\-cds.c}!save\_\-cds@{save\_\-cds}}
\index{save\_\-cds@{save\_\-cds}!sp_cds.c@{sp\_\-cds.c}}
\paragraph{\setlength{\rightskip}{0pt plus 5cm}void save\_\-cds (FILE $\ast$ {\em fp}, \/  {\bf sp\_\-cds} $\ast$ {\em A})}\hfill}
\label{sp__cds_8c_dbc8b3571c8ace5419d2e02318988388}


Imprime la matriz {\tt A} en formato matlab en el archivo {\tt fp}. 

\begin{Desc}
\item[Parámetros:]
\begin{description}
\item[{\em fp}]Archivo en donde imprimir \item[{\em A}]Matriz a imprimir, en formato CDS\end{description}
\end{Desc}
Esta función es útil para respaldar matrices. 

Definición en la línea 104 del archivo sp\_\-cds.c.

Hace referencia a binary\_\-search(), sp\_\-cds::dx, sp\_\-cds::ncol, sp\_\-cds::ndiag, sp\_\-cds::nrow, y sp\_\-cds::val.

Referenciado por bal\_\-save\_\-cds().

\begin{Code}\begin{verbatim}105 {
106     unsigned int i, j;
107     int diag, k;
108 
109     fprintf(fp, "[\n");
110 
111     for (i=0; i < A->nrow; ++i) {
112         for (j=0; j < A->ncol; ++j) {
113             diag = j - i;
114             k = binary_search(A->dx, A->ndiag, diag);
115             if (k != -1)
116                 fprintf(fp, " %g", A->val[k][i]);
117             else
118                 fprintf(fp, " 0");
119         }
120         if (i+1 < A->nrow)
121             fprintf(fp, ";\n");
122         else
123             fprintf(fp, "\n");
124     }
125 
126     fprintf(fp, "]\n");
127 }
\end{verbatim}
\end{Code}


\hypertarget{sp__cds_8c_1f5bfb80b511e70892a29e07d4769e98}{
\index{sp\_\-cds.c@{sp\_\-cds.c}!sp\_\-imprimir\_\-cds@{sp\_\-imprimir\_\-cds}}
\index{sp\_\-imprimir\_\-cds@{sp\_\-imprimir\_\-cds}!sp_cds.c@{sp\_\-cds.c}}
\paragraph{\setlength{\rightskip}{0pt plus 5cm}void sp\_\-imprimir\_\-cds (FILE $\ast$ {\em fp}, \/  {\bf sp\_\-cds} $\ast$ {\em mat})}\hfill}
\label{sp__cds_8c_1f5bfb80b511e70892a29e07d4769e98}


Imprime la matriz guardada en formato simple por coordenadas en fp. 

\begin{Desc}
\item[Parámetros:]
\begin{description}
\item[{\em fp}]Archivo en el cual se imprimirá la matriz \item[{\em mat}]Matriz a imprimir en formato CDS \end{description}
\end{Desc}


Definición en la línea 74 del archivo sp\_\-cds.c.

Hace referencia a sp\_\-cds::dx, sp\_\-cds::maxdiaglength, sp\_\-cds::ncol, sp\_\-cds::ndiag, sp\_\-cds::nrow, y sp\_\-cds::val.

Referenciado por bal\_\-imprimir\_\-cds().

\begin{Code}\begin{verbatim}75 {
76     unsigned int i, j;
77 
78     fprintf(fp, "Cantidad de filas: %d\n", mat->nrow);
79     fprintf(fp, "Cantidad de columnas: %d\n", mat->ncol);
80     fprintf(fp, "Cantidad de diagonales no-cero: %d\n", mat->ndiag);
81     fprintf(fp, "Largo maximo de diagonal: %d\n", mat->maxdiaglength);
82 
83     fprintf(fp, "Mapeo de diagonales:");
84     for (i=0; i < mat->ndiag; ++i)
85         fprintf(fp, " %d", mat->dx[i]);
86 
87     fprintf(fp, "\nValores:\n");
88     for (i=0; i < mat->ndiag; ++i) {
89         fprintf(fp, "val[%d,0:%d] =", i, mat->maxdiaglength-1);
90         for (j=0; j < mat->maxdiaglength; ++j)
91             fprintf(fp, " %g", mat->val[i][j]);
92         fprintf(fp, "\n");
93     }
94 }
\end{verbatim}
\end{Code}


