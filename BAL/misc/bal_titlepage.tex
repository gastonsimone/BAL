\begin{titlepage}
\vspace{1cm}

\begin{center}
\Huge{{\bf BAL:} Biblioteca de Álgebra Lineal Numérica}
\end{center}

\vspace{0.1cm}

\begin{center}
\Large{%
Gastón Simone, Pablo Ezzatti}
\end{center}

\vspace{0.1cm}

\begin{center}
\LARGE{Instituto de Computación - Facultad de Ingeniería \\
Universidad de la República \\
Montevideo, Uruguay}\\
\Large{%
\url{gaston.simone@gmail.com}\\
\url{pezzatti@find.edu.uy}}
\end{center}

\vspace{0.3cm}

\begin{center}
\Large{Junio, 2008}
\end{center}

\vspace{0.2cm}

\begin{center}
\begin{large}
\textbf{Palabras claves:} Álgebra Lineal Numérica, C, Algoritmos.
\end{large}
\end{center}

\begin{abstract}
\begin{large}
La resolución de una gran cantidad de modelos presentes en la computación científica se basan en la solución de
problemas del álgebra lineal numérica (ALN). Esta situación ha motivado fuertemente el desarrollo del área. En
contraposición al importante desarrollo se ha incrementado en forma abrupta la complejidad de las estrategias
de resolución, dificultando la comprensión por parte de los alumnos de los algoritmos utilizados.

En este contexto la propuesta se centra en el desarrollo de una biblioteca de ALN de carácter didáctico.
Por esta razón, la documentación es vasta y el código fue escrito pensando en su fácil lectura. Las rutinas
implementadas son eficientes desde el punto de vista algorítmico. Pero determinadas mejoras de desempeño,
propias de la implementación, fueron descartadas para mantener la legibilidad del código.

Este documento presenta el diseño, las funcionalidades y la implementación realizada.
\end{large}
\end{abstract}

\end{titlepage}

\endinput

